\section{Лекция 1. Введение в математическую логику}



\subsection{Классическое исчисление высказываний}




\deff{def:}
\textbf{Высказывание} (формула) строится по правилам:
\begin{itemize}
\item \textbf{Атомарное}: $A, B', C_{1234}$ (пропозициональные переменные)
\item \textbf{Составное}: если $\alpha$ и $\beta$ - высказывания, то:
\begin{itemize}
\item Отрицание: $(\neg\alpha)$
\item Конъюнкция: $(\alpha\with\beta)$ или $(\alpha\wedge\beta)$
\item Дизъюнкция: $(\alpha\vee\beta)$
\item Импликация: $(\alpha\rightarrow\beta)$ или $(\alpha\supset\beta)$
\end{itemize}
\end{itemize}


\begin{exmprus}
$$(((A\rightarrow B)\vee (B\rightarrow C)) \vee (C \rightarrow A))$$
\end{exmprus}


\deff{Метапеременные}: $\alpha, \beta, \gamma, \dots$ --- вместо них в формулу можно подставить, что угодно
 
 \textbf{Переменные для пропозициональных переменных}: $X, Y_n, Z'$
 
 \textbf{Приоритет связок}: отрицание, конъюнкция, дизъюнкция, импликация 
 
 \textbf{Ассоциативность}: левая для $\with$ и $\vee$, правая для $\rightarrow$







\deff{def:} \textbf{Оценка высказываний} определяется:
\begin{itemize}
\item Множество значений: $V = \{\textit{И},\textit{Л}\}$
\item Функция интерпретации: $f: \mathcal{P} \rightarrow V$,  где $\mathcal{P}$ - множество пропозициональных переменных.
\item Синтаксис оценки: $\llbracket \alpha \rrbracket^{X_1 := v_1,\ \dots,\ X_n := v_n}$
\end{itemize}



\begin{align*}
&\llbracket X \rrbracket = f(X)\\
&\llbracket X \rrbracket^{X := a} = a\\
&\llbracket \neg \alpha \rrbracket = 
  \begin{cases}
  \textit{Л}, & \text{если }\llbracket\alpha\rrbracket=\textit{И}\\
  \textit{И}, & \text{иначе}
  \end{cases}\\
&\llbracket \alpha \with \beta \rrbracket = 
  \begin{cases}
  \textit{И}, & \text{если }\llbracket\alpha\rrbracket=\llbracket\beta\rrbracket=\textit{И}\\ 
  \textit{Л}, & \text{иначе}
  \end{cases}\\
&\llbracket \alpha \vee \beta \rrbracket = 
  \begin{cases}
  \textit{Л}, & \text{если }\llbracket\alpha\rrbracket=\llbracket\beta\rrbracket=\textit{Л}\\
  \textit{И}, & \text{иначе}
  \end{cases}\\
&\llbracket \alpha \rightarrow \beta \rrbracket = 
  \begin{cases}
  \textit{Л}, & \text{если }\llbracket\alpha\rrbracket=\textit{И},\ \llbracket\beta\rrbracket=\textit{Л}\\
  \textit{И}, & \text{иначе}
  \end{cases}
\end{align*}



\deff{def:}
$\alpha$ - \textbf{тавтология} ($\models \alpha$), если истинна при всех оценках

\begin{exmprus}
$A\rightarrow A$ - тавтология, $A\rightarrow\neg A$ - не тавтология
\end{exmprus}

\deff{def:}
\begin{itemize}
\item $\gamma_1, \dots, \gamma_n \models \alpha$ - $\alpha$ \textbf{следствие}
\item \textbf{Выполнима} - истинна при некоторой оценке
\item \textbf{Невыполнима} - ложна при всех оценках
\item \textbf{Опровержима} - ложна при некоторой оценке
\end{itemize}


\subsection{Теория доказательств}

\deff{def:}
\textbf{Схема высказывания} - строка, где вместо переменных можно использовать метапеременные



\deff{def:}
Высказывание $\sigma$ \textbf{строится по схеме} $\textit{Ш}$, если
$$\sigma = \textit{Ш}[\textit{ч}_1 := \varphi_1][\textit{ч}_2 := \varphi_2]...[\textit{ч}_n := \varphi_n]$$



\deff{Схемы аксиом:}
\begin{enumerate}
\item $\alpha \rightarrow \beta \rightarrow \alpha$
\item $(\alpha \rightarrow \beta) \rightarrow (\alpha \rightarrow \beta \rightarrow \gamma) \rightarrow (\alpha \rightarrow \gamma)$
\item $\alpha \rightarrow \beta \rightarrow \alpha \& \beta$
\item $\alpha \& \beta \rightarrow \alpha$
\item $\alpha \& \beta \rightarrow \beta$
\item $\alpha \rightarrow \alpha \vee \beta$
\item $\beta \rightarrow \alpha \vee \beta$
\item $(\alpha \rightarrow \gamma) \rightarrow (\beta \rightarrow \gamma) \rightarrow (\alpha \vee \beta \rightarrow \gamma)$
\item $(\alpha \rightarrow \beta) \rightarrow (\alpha \rightarrow \neg \beta) \rightarrow \neg \alpha$
\item $\neg \neg \alpha \rightarrow \alpha$
\end{enumerate}

\deff{Правило вывода Modus Ponens}

\begin{itemize}
\item Формально: $$\infer{\beta}{\alpha & \alpha\rightarrow\beta}$$
\item Пример: <<Сейчас сентябрь; если сентябрь, то осень; следовательно, осень>>
\end{itemize}

\deff{def:}
\textbf{Доказательство} - последовательность $\delta_1, \dots, \delta_n$, где каждое $\delta_i$:
\begin{itemize}
\item Аксиома, или
\item Получено по MP из предыдущих
\end{itemize}


\deff{def:}
\textbf{Вывод из гипотез} $\Gamma$ - то же, но можно использовать гипотезы из $\Gamma$


\deff{def:}
 \textbf{Корректность}: $\vdash\alpha \Rightarrow \models\alpha$

\deff{def:} \textbf{Полнота}: $\models\alpha \Rightarrow \vdash\alpha$


\thmm{Теорема}

Исчисление высказываний корректно
\


\deff{Доказательство:}

Индукция по длине вывода + проверка аксиом и правила MP


\thmm{Теорема о дедукции}
$$\Gamma,\alpha\vdash\beta \Leftrightarrow \Gamma\vdash\alpha\rightarrow\beta$$

\textbf{Доказательство:}

Конструктивное доказательство: преобразование вывода с гипотезой $\alpha$ в вывод импликации $\alpha\rightarrow\beta$. Оно будет на следующей лекции

\textbf{def:} \deff{Формула доказуема}, если $\exists$ последовательность формул, каждая из которых это либо аксиома, либо получена правилом вывода и последней формулой является $\alpha$.
