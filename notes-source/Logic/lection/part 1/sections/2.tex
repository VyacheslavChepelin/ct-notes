\section{Лекция 2.}

\subsection{Теорема о дедукции}
Каковы бы ни были $\Gamma$, $\alpha$ и $\beta$:
$\Gamma,\alpha\vdash\beta$ выполнено тогда и только тогда, когда выполнено $\Gamma\vdash\alpha\rightarrow\beta$


Доказательство <<в две стороны>>, сперва <<справа налево>>.
Фиксируем $\Gamma,\alpha,\beta$.
Пусть $\Gamma\vdash\alpha\rightarrow\beta$, покажем $\Gamma,\alpha\vdash\beta$

То есть по условию существует вывод: $$\delta_1, \delta_2, \dots, \delta_{n-1}, \alpha\rightarrow\beta$$

Тогда следующая последовательность --- тоже вывод: 
$$\delta_1, \delta_2, \dots, \delta_{n-1}, \alpha\rightarrow\beta, \alpha, \beta$$

\deff{Доказательство:} 

Покажем, что $\Gamma\vdash\alpha\rightarrow\beta$ влечёт $\Gamma,\alpha\vdash\beta$

\begin{tabular}{lll}
№ п/п & формула & пояснение\\
\hline
$(1)$ & $\delta_1$ & в соответствии с исходным доказательством\\
    & $\dots$ \\
$(n-1)$ & $\delta_{n-1}$ & в соответствии с исходным доказательством\\
$(n)$ & $\alpha\rightarrow\beta $ & в соответствии с исходным доказательством\\
$(n+1)$ & $\alpha$ & гипотеза\\
$(n+2)$ & $\beta$ & Modus Ponens n$+1$, $n$
\end{tabular}

Вывод $\Gamma,\alpha\vdash\beta$ предоставлен, первая часть теоремы доказана.

Покажем, что $\Gamma,\alpha\vdash\beta$ влечёт $\Gamma\vdash\alpha\rightarrow\beta$:

Пусть даны формулы вывода $$\delta_1,\delta_2,\dots,\delta_{n-1},\beta$$

Аналогично предыдущему пункту, перестроим вывод.

Построим <<черновик>> вывода, приписав $\alpha$ слева к каждой формуле:
$$\alpha\rightarrow\delta_1,\alpha\rightarrow\delta_2,\dots,\alpha\rightarrow\delta_{n-1},\alpha\rightarrow\beta$$

Данная последовательность формул не обязательно вывод: $\Gamma:=\varnothing$, $\alpha := A$
$$\delta_1 := A\rightarrow B\rightarrow A$$

припишем $A$ слева --- вывод не получим:
$$\alpha\rightarrow\delta_1 \equiv A \rightarrow (A\rightarrow B\rightarrow A)$$




\deff{def:} \textbf{конечная последовательность}
 --- это функция $\delta: 1\dots n \rightarrow \mathcal{F}$


\deff{def:} \textbf{Кон. последовательность, индексированная дробными числами} --- это
функция $\zeta: I \rightarrow \mathcal{F}$, где $I \subset \mathbb{Q}$ и $I$ конечно.

Продолжим доказательство: $\Gamma,\alpha\vdash\beta$ влечёт $\Gamma\vdash\alpha\rightarrow\beta$:

Будем делать индукцию по длине вывода: Если $\delta_1, \dots, \delta_n$ --- вывод
$\Gamma,\alpha\vdash\delta_n$, то найдётся вывод $\zeta_k$ для $\Gamma\vdash\alpha\rightarrow\delta_n$,
причём $\zeta_1 \equiv \alpha\rightarrow\delta_1, \dots, \zeta_n \equiv \alpha\rightarrow\delta_n$.

\begin{itemize}
\item База $(n=1)$: частный случай перехода (без M.P.).

\item Переход. Пусть $\delta_1, \dots, \delta_{n+1}$ --- исходный вывод. И пусть (по индукционному предположению)
уже по начальному фрагменту $\delta_1, \dots, \delta_n$ построен вывод $\zeta_k$ утверждения 
$\Gamma\vdash\alpha\rightarrow\delta_n$. 

Но $\delta_{n+1}$ как-то был обоснован --- разберём случаи:
\begin{enumerate}
\item $\delta_{n+1}$ --- аксиома или $\delta_{n+1} \in \Gamma$ 
\item $\delta_{n+1}\equiv\alpha$
\item $\delta_{n+1}$ --- Modus Ponens из $\delta_j$ и 
$\delta_k \equiv \delta_j\rightarrow\delta_{n+1}$.
\end{enumerate}

В каждом из случаев можно дополнить черновик до полноценного вывода.
\end{itemize}


Случай аксиомы (продолжение):

\begin{tabular}{lll}
№ п/п & новый вывод & пояснение \\
\hline
& \dots\\
$(1)$ & $\alpha\rightarrow\delta_1$ \\
& \dots\\
$(2)$ & $\alpha\rightarrow\delta_2$ \\
    & \dots \\
\color{cyan}$(n+0.3)$ & \color{cyan}$\delta_{n+1}\rightarrow\alpha\rightarrow\delta_{n+1}$ & \color{cyan}схема аксиом 1\\
\color{cyan}$(n+0.6)$ & \color{cyan}$\delta_{n+1}$ & \color{cyan}аксиома, либо $\delta_{n+1} \in \Gamma$\\
$(n+1)$ & $\alpha\rightarrow\delta_{n+1}$ & M.P. $n+0.6$, $n+0.3$\\
\end{tabular}

Случай $\delta_{n+1}\equiv\alpha$:

\begin{tabular}{lll}
№ п/п & новый вывод & пояснение \\
\hline
    & \dots \\
$(1)$ & $\alpha\rightarrow\delta_1$ \\
    & \dots \\
$(2)$ & $\alpha\rightarrow\delta_2$ \\
    & \dots \\
\color{cyan}$(n+0.2)$ & \color{cyan}$\alpha \rightarrow (\alpha \rightarrow \alpha)$ & \color{cyan} Сх. акс. 1\\
\color{cyan}$(n+0.4)$ & \color{cyan}$(\alpha \rightarrow (\alpha \rightarrow \alpha)) \rightarrow 
  (\alpha \rightarrow (\alpha \rightarrow \alpha) \rightarrow \alpha) \rightarrow
  (\alpha \rightarrow \alpha)$& \color{cyan}Сх. акс. 2\\
\color{cyan}$(n+0.6)$ & \color{cyan}$(\alpha \rightarrow (\alpha \rightarrow \alpha) \rightarrow \alpha) \rightarrow
  (\alpha \rightarrow \alpha)$ &\color{cyan}M.P. $n+0.2$, $n+0.4$\\
\color{cyan}$(n+0.8)$ & \color{cyan}$\alpha \rightarrow (\alpha \rightarrow \alpha) \rightarrow \alpha$ & 
    \color{cyan}Сх. акс. 1\\
$(n+1)$ & $\alpha \rightarrow \alpha$ & M.P. $n+0.8$, $n+0.6$\\
\end{tabular}

Случай Modus Ponens:

\begin{tabular}{lll}
№ п/п & новый вывод & пояснение \\
\hline
    & \dots \\
$(1)$ & $\alpha\rightarrow\delta_1$ \\
    & \dots \\
$(2)$ & $\alpha\rightarrow\delta_2$ \\
    & \dots \\
$(j)$ & $\alpha\rightarrow\delta_j$ \\
    & \dots \\
$(k)$ & $\alpha\rightarrow\delta_j\rightarrow\delta_{n+1}$ \\
    & \dots \\
\color{cyan}$(n+0.3)$ & \color{cyan}$(\alpha\rightarrow\delta_j)
    \rightarrow(\alpha\rightarrow\delta_j\rightarrow\delta_{n+1})\rightarrow(\alpha\rightarrow\delta_{n+1})$ & \color{cyan}Сх. акс. 2\\
\color{cyan}$(n+0.6)$ & \color{cyan}$(\alpha\rightarrow\delta_j
    \rightarrow\delta_{n+1})\rightarrow(\alpha\rightarrow\delta_{n+1})$ & \color{cyan}M.P. $j$, $n+0.3$\\
$(n+1)$ & $\alpha\rightarrow\delta_{n+1}$ & M.P. $k$, $n+0.6$\\
\end{tabular}

\hfill Q.E.D.

\deff{Некоторые полезные правила}
\begin{enumerate}
    \item  \textbf{Правило контрапозиции.}
$\vdash (\alpha \rightarrow \beta) \rightarrow (\neg\beta \rightarrow \neg\alpha)$.
    \item  \textbf{Правило исключённого третьего.} $\vdash\alpha\vee\neg\alpha$.
    \item  \textbf{Об исключении допущения}
Пусть справедливо $\Gamma, \rho \vdash \alpha$ и $\Gamma, \neg \rho \vdash \alpha$.
Тогда также справедливо $\Gamma \vdash \alpha$.
\end{enumerate}








\subsection{Теорема о полноте исчисления высказываний}

\thmm{Теорема.} Если $\models\alpha$, то $\vdash\alpha$.


\deff{def:} \textbf{условное отрицание}
Зададим некоторую оценку переменных, такую, что $\llbracket\alpha\rrbracket = x$. 

Тогда \emph{условным отрицанием} формулы $\alpha$ назовём следующую формулу $\llparenthesis\alpha\rrparenthesis$:
$$\llparenthesis\alpha\rrparenthesis = \left\{\begin{array}{ll}\alpha, & x = \textnormal{И}\\
       \neg\alpha, & x = \textnormal{Л}\end{array}\right.$$


Аналогично записи для оценок, будем указывать оценку переменных, если это потребуется / будет неочевидно из контекста:
$$\llparenthesis \neg X \rrparenthesis^{X:=\textnormal{Л}} = \neg X\quad\quad\quad\llparenthesis \neg X \rrparenthesis^{X:=\textnormal{И}} = \neg\neg X$$

Также, если $\Gamma := \gamma_1, \gamma_2, \dots, \gamma_n$, то за $\llparenthesis \Gamma \rrparenthesis$ 
обозначим $\llparenthesis \gamma_1 \rrparenthesis, \llparenthesis \gamma_2 \rrparenthesis, \dots \llparenthesis \gamma_n \rrparenthesis$.

\deff{Таблицы истинности и высказывания}

Рассмотрим связку <<импликация>> и её таблицу истинности:

\begin{center}\begin{tabular}{cccc}
$\llbracket A\rrbracket$ & $\llbracket B\rrbracket$ & $\llbracket A\rightarrow B\rrbracket$ & формула\\\hline
Л & Л & И & $\neg A, \neg B \vdash A \rightarrow B$\\
Л & И & И & $\neg A, B \vdash A \rightarrow B$\\
И & Л & Л & $A, \neg B \vdash \neg (A \rightarrow B)$\\
И & И & И & $A, B \vdash A \rightarrow B$
\end{tabular}\end{center}

Заметим, что с помощью условного отрицания данную таблицу можно записать в одну строку:
$$\llparenthesis A \rrparenthesis, \llparenthesis B \rrparenthesis \vdash \llparenthesis A \rightarrow B \rrparenthesis $$

\thmm{Теорема (О полноте исчисления высказываний)}
Если $\models\alpha$, то $\vdash\alpha$


\begin{enumerate}
\item Построим таблицы истинности для каждой связки $(\star)$ и докажем в них каждую строку:
$$ \llparenthesis\varphi\rrparenthesis, \llparenthesis\psi\rrparenthesis \vdash \llparenthesis\varphi\star\psi\rrparenthesis$$
\item Построим таблицу истинности для $\alpha$ и докажем в ней каждую строку:
$$\llparenthesis \Xi \rrparenthesis \vdash \llparenthesis \alpha \rrparenthesis$$
\item Если формула общезначима, то в ней все строки будут иметь вид $\llparenthesis \Xi \rrparenthesis \vdash\alpha$,
потому от гипотез мы сможем избавиться и получить требуемое $\vdash\alpha$.
\end{enumerate}

\deff{Доказательство:}

\deff{Шаг 1. Лемма о связках}

Запись
$$\llparenthesis\varphi\rrparenthesis, \llparenthesis\psi\rrparenthesis \vdash \llparenthesis\varphi\star\psi\rrparenthesis$$
сводится к 14 утверждениям:

\begin{center}\begin{tabular}{rclp{1cm}rcl}
$\neg\varphi, \neg\psi$&$ \vdash $&$\neg (\varphi \with \psi)$& & $\neg\varphi, \neg\psi$&$ \vdash $&$     (\varphi \rightarrow  \psi)$ \\
$\neg\varphi,     \psi$&$ \vdash $&$\neg (\varphi \with \psi)$& &$\neg\varphi,     \psi$&$ \vdash $&$     (\varphi \rightarrow  \psi)$ \\
$    \varphi, \neg\psi$&$ \vdash $&$\neg (\varphi \with \psi)$& &$ \varphi, \neg\psi$&$ \vdash $&$\neg (\varphi \rightarrow  \psi)$ \\
$    \varphi,     \psi$&$ \vdash $&$     (\varphi \with \psi)$& &$    \varphi,     \psi$&$ \vdash $&$     (\varphi \rightarrow  \psi)$ \\
$\neg\varphi, \neg\psi$&$ \vdash $&$\neg (\varphi \vee  \psi)$& &$    \varphi          $&$ \vdash $&$     \neg\neg\varphi$ \\
$\neg\varphi,     \psi$&$ \vdash $&$     (\varphi \vee  \psi)$& &$\neg\varphi          $&$ \vdash $&$         \neg\varphi$\\
$    \varphi, \neg\psi$&$ \vdash $&$     (\varphi \vee  \psi)$ \\
$    \varphi,     \psi$&$ \vdash $&$     (\varphi \vee  \psi)$
\end{tabular}\end{center}

\deff{Шаг 2. Обобщение на любую формулу}

\deff{Лемма (Условное отрицание формул)}
Пусть пропозициональные переменные $\Xi := \{X_1, \dots, X_n\}$ ---
все переменные, которые используются в формуле $\alpha$. И пусть
задана некоторая оценка переменных.

Тогда, $\llparenthesis \Xi \rrparenthesis \vdash\llparenthesis\alpha\rrparenthesis$

\deff{Доказательстсво леммы:}

Индукция по длине формулы $\alpha$.
\begin{itemize}
\item База: формула $\alpha$ --- атомарная, т.е. $\alpha \equiv X_i$. Тогда при любом $\Xi$ выполнено 
$\llparenthesis\Xi\rrparenthesis^{X_i := \text{И}} \vdash X_i$ и $\llparenthesis\Xi\rrparenthesis^{X_i := \text{Л}} \vdash \neg X_i$.
\item Переход: $\alpha \equiv \varphi\star\psi$, причём $\llparenthesis\Xi\rrparenthesis\vdash\llparenthesis\varphi\rrparenthesis$
и $\llparenthesis\Xi\rrparenthesis\vdash\llparenthesis\psi\rrparenthesis$

Тогда построим вывод: 

\begin{tabular}{lll}
$(1)\dots(n)$ & $\llparenthesis\varphi\rrparenthesis$ & индукционное предположение\\
$(n+1)\dots(k)$ & $\llparenthesis\psi\rrparenthesis$ & индукционное предположение\\
$(k+1)\dots(l)$ & $\llparenthesis\varphi\star\psi\rrparenthesis$ & 
  лемма о связках: $\llparenthesis\varphi\rrparenthesis$ и $\llparenthesis\psi\rrparenthesis$ доказаны выше,\\
  & & значит, их можно использовать как гипотезы
\end{tabular}
\end{itemize}

\hfill Q.E.D. Леммы

\deff{Шаг 3. Избавляемся от гипотез}

\begin{lemmarus}Пусть при всех оценках переменных
$\llparenthesis\Xi\rrparenthesis \vdash \alpha$, тогда
$\vdash\alpha$.
\end{lemmarus}

\textbf{Доказательство:}

Индукция по количеству переменных $n$.

\begin{itemize}
\item База: $n=0$. Тогда $\vdash\alpha$ есть из условия.
\item Переход: пусть $\llparenthesis X_1, X_2,  \dots X_{n+1} \rrparenthesis \vdash \alpha$.
Рассмотрим $2^n$ пар выводов: $$\llparenthesis X_1, X_2, \dots X_n\rrparenthesis,X_{n+1} \vdash \alpha\quad\quad\llparenthesis X_1, X_2, \dots X_n\rrparenthesis,\neg X_{n+1} \vdash \alpha$$
По лемме об исключении допущения тогда
$$\llparenthesis X_1, X_2, \dots X_n \rrparenthesis \vdash \alpha$$
\end{itemize}
При этом, $\llparenthesis X_1, X_2, \dots X_n \rrparenthesis  \vdash \alpha$ при всех оценках
переменных $X_1, \dots X_n$. Значит, $\vdash\alpha$ по индукционному предположению.

\hfill Q.E.D. Леммы

\textbf{Замечание:}

Теорема о полноте --- конструктивна. Получающийся вывод --- экспоненциальный по длине.

Несложно по изложенному доказательству разработать программу, строящую вывод.

Вывод для формулы с 3 переменными --- порядка 3 тысяч строк.

\deff{def:} Полная теория - в которой выполняется теорема о полноте


\newpage 

\subsection{Интуиционистская логика}







Основные положения \deff{интуционизма}:
\begin{enumerate}
\item Математика не формальна.
\item Математика независима от окружающего мира.
\item Математика не зависит от логики — это логика зависит от математики.
\end{enumerate}

То есть суть в том, что мы доказываем, что какой-то объект существует <<на самом деле>>, не как в теореме о неподвижной точке например (там мы просто показываем, что такой точки не может не быть, но есть ли она?)

\deff{BHK-интерпретация логических связок}

BHK — это сокращение трёх фамилий: Брауэр, Гейтинг, Колмогоров.

Пусть $\alpha$, $\beta$ --- некоторые конструкции, тогда:

\begin{itemize}
\item $\alpha\ \&\ \beta$ построено, если построены $\alpha$ и $\beta$
\item $\alpha \vee \beta$ построено, если построено $\alpha$ или $\beta$,
и мы знаем, что именно
\item $\alpha\rightarrow\beta$ построено, если есть способ перестроения
$\alpha$ в $\beta$
\item $\bot$ — конструкция, не имеющая построения
\item $\neg\alpha$ построено, если построено $\alpha\rightarrow\bot$
\end{itemize}

\deff{Дизъюнкция}

Конструкция $\alpha\vee\neg\alpha$ не имеет построения в общем случае.
Что может быть построено: $\alpha$ или $\neg\alpha$?

Возьмём за $\alpha$ нерешённую проблему, например, $P = NP$

Авторам в данный момент не известно, выполнено $P = NP$ или же $P \ne NP$.

\deff{Отличия импликации}

Высказывание общезначимо в И.В. и не выполнено в BHK-интерпретации:
$$(A \rightarrow B) \vee (B \rightarrow C) \vee (C \rightarrow A)$$

Давайте дадим следующий смысл пропозициональным переменным:
\begin{itemize}
\item $A$ --- 13.09.2025 в Санкт-Петербурге идёт дождь;
\item $B$ --- 13.09.2025 в Санкт-Петербурге светит солнце;
\item $C$ --- во 2 семестре ровно 2 человека из групп 38-39 получили <<отлично>> по матанализу, списав.
\end{itemize}

Импликацию можно понимать как <<формальную>> и как <<материальную>>.
\begin{itemize}
\item Материальная импликация $A\rightarrow B$ --- надо посмотреть в окно.

\item Формальная импликация $A\rightarrow B$ места не имеет (причинно-следственной связи нет).
\end{itemize}

Формализация интуиционистской логики возможна, но интуитивное понимание --- основное.

\deff{def:} Аксиоматика интуиционистского исчисления высказываний в гильбертовском стиле: 
аксиоматика КИВ, в которой 10 схема аксиом

\begin{center}\begin{tabular}{ll}
(10) & $\neg \neg \alpha \rightarrow \alpha$
\end{tabular}\end{center}

заменена на 

\begin{center}\begin{tabular}{ll}
(10и) & $\alpha \rightarrow \neg\alpha \rightarrow \beta$
\end{tabular}\end{center}


\deff{Интуиционистское И.В. (натуральный, естественный вывод)}
\begin{itemize}
\item Формулы языка (секвенции) имеют вид: $\Gamma\vdash\alpha$.
Правила вывода: 
\begin{flushright}$\quad\quad\quad\infer[(\text{аннотация})]{\text{заключение}}{\text{посылка 1}\quad\quad\text{посылка 2}\quad\quad\dots}$\end{flushright}
\item Аксиома:\\$\infer[\text{(акс.)}]{\Gamma,\alpha\vdash\alpha}{\vphantom{\Gamma}}$ 

\item Правила введения связок:\\$\infer{\Gamma\vdash\alpha\rightarrow\beta}{\Gamma,\alpha\vdash\beta}\quad\quad\infer{\Gamma\vdash\alpha\vee\beta}{\Gamma\vdash\alpha}$, $\infer{\Gamma\vdash\alpha\vee\beta}{\Gamma\vdash\beta}\quad\quad\infer{\Gamma\vdash\alpha\with\beta}{\Gamma\vdash\alpha\quad\quad\Gamma\vdash\beta}$

\item Правила удаления связок:\\$\infer{\Gamma\vdash\beta}{\Gamma\vdash\alpha\quad\Gamma\vdash\alpha\rightarrow\beta}\quad\quad\infer{\Gamma\vdash\gamma}{\Gamma\vdash\alpha\rightarrow\gamma\quad\Gamma\vdash\beta\rightarrow\gamma\quad\Gamma\vdash\alpha\vee\beta}$
 $\infer{\Gamma\vdash\alpha}{\Gamma\vdash\alpha\with\beta}\quad\quad\infer{\Gamma\vdash\beta}{\Gamma\vdash\alpha\with\beta}\quad\quad\infer{\Gamma\vdash\alpha}{\Gamma\vdash\bot}$
\item Пример доказательства:\vspace{-0.3cm}
$$\infer[(\text{введ}\with)]{A\with B\vdash B \with A}{\infer[(\text{удал}\with)]{A \with B \vdash B}{\infer[(\text{акс.})]{A \with B\vdash A \with B}{}}
                                           \quad\quad\infer[(\text{удал}\with)]{A \with B \vdash A}{\infer[(\text{акс.})]{A \with B\vdash A \with B}{}}}$$
\end{itemize}


Откуда теперь у нас есть Гильбертов вывод (10 аксиом, по порядку), а есть натуральный (1 аксиома, правила введения и удаления связок, в виде дерева)


\newpage 

\subsection{Топологическое пространство}

\deff{def:} \textbf{Топологическим пространством} называется упорядоченная пара $\langle X, \Omega \rangle$,
где $X$ --- некоторое множество, а $\Omega \subseteq \mathcal{P}(X)$, причём:
\begin{enumerate}
\item $\varnothing, X \in \Omega$
\item если $A_1, \dots, A_n \in \Omega$, то $A_1 \cap A_2 \cap \dots \cap A_n \in \Omega$;
\item если $\{A_\alpha\}$ --- семейство множеств из $\Omega$, то и $\bigcup_\alpha A_\alpha \in \Omega$.
\end{enumerate}

Множество $\Omega$ называется \textbf{топологией}.
Элементы $\Omega$ называются \textbf{открытыми множествами.}

\deff{def:} \textbf{Внутренность множества} $A^\circ$ --- наибольшее $T$, что $T \in \Omega$ и $T \subseteq A$. 

\deff{def:} Множество \textbf{замкнуто}, если дополнение открыто.

\deff{Топологические пространства как модель ИИВ}

\thmm{Теорема.} Если $\langle X, \Omega\rangle$ --- некоторое топологическое пространство, то следующий способ оценки высказываний
даёт корректную модель ИИВ: $V = \Omega$, $\text{и} = X$ и 
\begin{tabular}{l}
$\llbracket \alpha \with \beta \rrbracket = \llbracket \alpha \rrbracket \cap \llbracket \beta \rrbracket$\\
$\llbracket \alpha \vee \beta \rrbracket = \llbracket \alpha \rrbracket \cup \llbracket \beta \rrbracket$\\
$\llbracket \neg\alpha \rrbracket = (c\llbracket \alpha \rrbracket)^\circ$\\
$\llbracket \alpha \rightarrow \beta \rrbracket = (c\llbracket \alpha \rrbracket \cup \llbracket \beta \rrbracket)^\circ$
\end{tabular}

\deff{def:}
$\models\alpha$ в топологических моделях, если при всех $\langle X,\Omega\rangle$ имеет место $\llbracket \alpha \rrbracket = X$.

\thmm{Теорема.}
Полнота топологических моделей ИИВ: $\models\alpha$ тогда и только тогда, когда $\vdash_\text{и}\alpha$.



\deff{Пример топологий:}

\begin{enumerate}
    \item \deff{Топология стрелки} - отрезки $[a, +\inf)$
    \item  \deff{Топология Зарисского} - $\{R, \varnothing\} \cup (X \subseteq \mathbb{R} | X^c - \text{конечно})$
    
\end{enumerate}

Топологии будут еще позже