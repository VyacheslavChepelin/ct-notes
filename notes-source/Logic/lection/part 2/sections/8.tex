
\section{Лекция 8.}

\subsection{Арифметизация в работах Лейбница}

\begin{itemize}
\item Любой термин --- пара взаимно простых чисел $+a-b$. Например, мудрый --- $+70-33$, благочестивый --- $+10-3$.

\item Общеутвердительное предложение (каждый $+a-b$ есть $+c-d$): $a : c$ и $b : d$.\\
Всякий мудрый есть благочестивый ($70 = 10\cdot 7$, $33 = 3 \cdot 11$).

\item Частноотрицательное предложение --- не верно общеутвердительное.

\item Общеотрицательное предложение --- когда $a,d$ или $b,c$ имеют общий делитель, отличный от 1:\\
Ни один благочестивый ($+10-3$) не есть несчастный ($+5-14$), так как $10=2 \cdot 5$ и $14 =2 \cdot 7$.
\end{itemize}

\subsection{Соглашения о записи}

\begin{itemize}
\item Рассматриваем функции $\mathbb{N}^n_0\to\mathbb{N}_0$.
\item Обозначим вектор $\langle x_1, x_2, \dots, x_n\rangle$ как $\overrightarrow{x}$.
\end{itemize}

\deff{Примитивы Z, N, U, S} 
\begin{enumerate}
\item Примитив <<Ноль>> ($Z$) 
$$Z: \mathbb{N}_0\to\mathbb{N}_0,\ \ \ \ \ Z(x_1) = 0$$

\item Примитив <<Инкремент>> ($N$) 
$$N: \mathbb{N}_0\to\mathbb{N}_0,\ \ \ \ \ N(x_1) = x_1+1$$

\item Примитив <<Проекция>> ($U$) — семейство функций; пусть $k,n \in \mathbb{N}_0, k \le n$
$$U^k_n: \mathbb{N}^n_0 \to \mathbb{N}_0,\ \ \ \ \ U^k_n(\overrightarrow{x}) = x_k$$

\item Примитив <<Подстановка>> ($S$) --- семейство функций; пусть $g: \mathbb{N}^k_0 \to \mathbb{N}_0,\ \ f_1,\dots,f_k: \mathbb{N}^n_0 \to \mathbb{N}_0$
$$S\langle g,f_1,f_2,\dots,f_k \rangle (\overrightarrow{x}) = g(f_1(\overrightarrow{x}),\dots,f_k(\overrightarrow{x}))$$
\end{enumerate}


\deff{Примитив <<примитивная рекурсия>>, $R$}

Пусть $f: \mathbb{N}^n_0\to\mathbb{N}_0$ и $g: \mathbb{N}^{n+2}_0 \to\mathbb{N}_0$.
Тогда $R\langle f,g\rangle: \mathbb{N}^{n+1}_0\to\mathbb{N}_0$, причём

$$R\langle f,g\rangle(\overrightarrow{x},y)=
 \left\{\begin{array}{ll} 
  f(\overrightarrow{x}), &y=0\\
  g(\overrightarrow{x},y-1,R\langle f,g\rangle (\overrightarrow{x},y-1)), &y > 0
\end{array}\right.$$

$$\begin{array}{ll}
    R\langle f,g\rangle(\overrightarrow{x},3) &= g(\overrightarrow{x},2,R\langle f,g\rangle(\overrightarrow{x},2)) \\
 &=   g(\overrightarrow{x},2,g(\overrightarrow{x},1,R\langle f,g\rangle(\overrightarrow{x},1))) \\
 &=   g(\overrightarrow{x},2,g(\overrightarrow{x},1,g(\overrightarrow{x},0,R\langle f,g\rangle(\overrightarrow{x},0)))) \\
 &=  g(\overrightarrow{x},2,g(\overrightarrow{x},1,g(\overrightarrow{x},0,f(\overrightarrow{x}))))
\end{array}$$

\deff{def:} Функция $f$ --- \deff{примитивно-рекурсивна}, если может быть выражена как композиция примитивов $Z$, $N$, $U$, $S$ и $R$.

\subsection{Примитивно-рекурсивные функции: $x+y$}

\textbf{Лемма:}
$f(a,b) = a+b$ примитивно-рекурсивна

\textbf{Доказательство:}

$f = R \langle U^1_1, S\langle N, U^3_3\rangle \rangle$:

$R\langle f,g\rangle(x,y)
=\left\{\begin{array}{ll} 
  f(x), &y=0\\
  g(x,y-1,R\langle f,g\rangle (x,y-1)), &y > 0
\end{array}\right.$ 

\begin{itemize}
\item База. $R\langle U^1_1, S\langle N, U^3_3\rangle \rangle(x,0) = U^1_1(x) = x$

\item Переход. $R\langle U^1_1, S\langle N, U^3_3\rangle \rangle(x,y+1) = $

$... = S\langle N, U^3_3\rangle (x,y,R\langle U^1_1(x), S\langle N, U^3_3\rangle\rangle (x,y) ) =$

$... = S\langle N, U^3_3\rangle (x,y,x + y) = $ 

$... = N(x+y) = x+y+1$
\end{itemize}

\hfill Q.E.D.

Какие функции примитивно-рекурсивные?

\begin{enumerate}
\item Сложение, вычитание
\item Умножение, деление
\item Вычисление простых чисел
\item Неформально: все функции, вычисляемые конечным числом вложенных циклов \verb!for!:

\begin{verbatim}
for (int i1 = 0; i1 < g1(x1...xn); i1++) {
    for (int i2 = 0; i2 < g2(x1...xn,i1); i2++) {
        ...
           for (int ik = 0; ik < gk(x1...xn,i1,i2...); ik++) {
               // выражение без циклов
           }
        ...
    }
}
\end{verbatim}
\end{enumerate}

\subsection{Общерекурсивные функции}

\deff{def:}
Функция --- \textbf{общерекурсивная}, если может быть построена при помощи примитивов $Z$, $N$, $U$, $S$, $R$ и примитива минимизации:
$$M\langle f \rangle (x_1,x_2,\dots,x_n) = \min\{y: f(x_1,x_2,\dots,x_n,y) = 0\}$$
Если $f(x_1,x_2,\dots,x_n,y) > 0$ при любом $y$, результат не определён.




\textbf{Пример:} Пусть $f(x,y) = x-y^2$, тогда $\lceil\sqrt{x}\rceil = M\langle f\rangle (x)$

\begin{verbatim}
int sqrt(int x) {
    int y = 0;
    while (x-y*y > 0) y++;
    return y;
}
\end{verbatim}

\deff{def:} Функция Аккермана:
$$A(m,n) = \left\{\begin{array}{ll}
  n+1,&m = 0\\
  A(m-1,1),&m > 0, n = 0\\
  A(m-1,A(m,n-1)),&m > 0, n > 0
\end{array}\right.$$



\thmm{Теорема.} 

Пусть $f(\overrightarrow{x})$ --- примитивно-рекурсивная. Тогда найдётся $k$, что $f(\overrightarrow{x}) < A(k,\max(\overrightarrow{x}))$

\textbf{Доказательство:}

Индукция по структуре $f$.
\begin{enumerate}
\item $f = Z$, тогда $k = 0$, т.к. $A(0,x) = x+1 > Z(x) = 0$;
\item $f = N$, тогда $k = 1$, т.к. $A(1,x) = x + 2 > N(x) = x+1$;
\item $f = U^n_s$, тогда $k = 0$, т.к. $f(\overrightarrow{x}) \le \max(\overrightarrow{x}) < A(0,\max(\overrightarrow{x}))$;
\item $f = S\langle g,h_1,\dots,h_n\rangle$, тогда $k = k_g + \max(k_{h_1},\dots,k_{h_n}) + 2$;
\item $f = R\langle g,h \rangle$, тогда $k = \max(k_g,k_h)+2$.
\end{enumerate}

\hfill Q.E.D.

\textbf{Лемма:}
Пусть $f = R\langle g,h \rangle$. Тогда при $k = \max(k_g,k_h)+2$ выполнено $f(\overrightarrow{x},y) \le A^{(y+1)}(k-2,\max(\overrightarrow{x},y))$.


\textbf{Доказательство:}

Индукция по $y$.
\begin{itemize}
\item База: $y = 0$. Тогда: $f(\overrightarrow{x},0) = g(\overrightarrow{x}) \le 
A(k_g,\max(\overrightarrow{x})) \le A^{(1)}(k-2,\max(\overrightarrow{x},0))$.

\item Переход: пусть $f(\overrightarrow{x},y) \le A^{(y+1)}(k-2,\max(\overrightarrow{x},y))$.
Тогда $f(\overrightarrow{x},y+1) = h(\overrightarrow{x},y,f(\overrightarrow{x},y))
\le A(k_h,\max(\overrightarrow{x},y,f(\overrightarrow{x},y))) 
\le A(k_h,\max(\overrightarrow{x},y,A^{(y+1)}(k-2,\max(\overrightarrow{x},y))) = A(k_h,A^{(y+1)}(k-2,\max(\overrightarrow{x},y)))
\le A^{(y+2)}(k-2,\max(\overrightarrow{x},y+1))$
\end{itemize}

\hfill Q.E.D.

Заметим, что $A^{(y+1)}(k-2,\max(\overrightarrow{x},y)) \le
A^{(\max(\overrightarrow{x},y)+1)}(k-2,\max(\overrightarrow{x},y)) \le
 A^{(\max(\overrightarrow{x},y)+2)}(k-2,\max(\overrightarrow{x},y)) <
A(k,\max(\overrightarrow{x},y))$

\subsection{Тезис Чёрча}

Тезис Чёрча для общерекурсивных функций: любая эффективно-вычислимая функция $\mathbb{N}^k_0\to\mathbb{N}_0$ является общерекурсивной.


\deff{def:}
Запись вида $\psi(\theta_1,\dots,\theta_n)$ означает $\psi[x_1:=\theta_1,\dots,x_n:=\theta_n]$

\deff{def:}\textbf{Литерал числа}
$$\overline{a} = \left\{\begin{array}{ll} 0, &\mbox{если } a = 0\\
                (\overline{b})', &\mbox{если } a = b+1
\end{array}\right.$$


Пример: пусть $\psi := x_1 = 0$. Тогда $\psi(\overline{3})$ соответствует формуле $0''' = 0$

\subsection{Выразимость отношений в Ф.А.}

\deff{def:} Будем говорить, что отношение $R\subseteq \mathbb{N}^n_0$ \textbf{выразимо в ФА}, если существует формула $\rho$, что:
\begin{enumerate}
\item если $\langle a_1,\dots,a_n \rangle \in R$, то $\vdash \rho(\overline{a_1},\dots,\overline{a_n})$
\item если $\langle a_1,\dots,a_n \rangle \notin R$, то $\vdash \neg\rho(\overline{a_1},\dots,\overline{a_n})$
\end{enumerate}

\thmm{Теорема} 

Отношение <<равно>> выразимо в Ф.А.: $R = \{ \langle x,x \rangle\ |\ x \in \mathbb{N}_0 \}$

\textbf{Доказательство:}

Пусть $\rho := x_1=x_2$. Тогда:
\begin{itemize}
\item $\vdash p = p$ при $p := \overline{k}$ при всех $k \in \mathbb{N}_0$: $\vdash 0=0$, $\vdash 0'=0'$, $\vdash 0''=0''$, ...
\item $\vdash \neg p = q$ при $p := \overline{k}$, $q := \overline{s}$ при всех $k,s \in \mathbb{N}_0$ и $k \ne s$.

$\vdash \neg 0 = 0'$, $\vdash \neg 0 = 0''$, $\vdash \neg 0''' = 0'$, ...
\end{itemize}

\hfill Q.E.D.

\subsection{Представимость функций в Ф.А.}

\deff{def:} Будем говорить, что функция $f: \mathbb{N}^n_0\to\mathbb{N}_0$ представима в ФА, если существует формула $\varphi$, что:
\begin{enumerate}
\item если $f(a_1,\dots,a_n) = u$, то $\vdash \varphi(\overline{a_1},\dots,\overline{a_n},\overline{u})$
\item если $f(a_1,\dots,a_n) \ne u$, то $\vdash \neg\varphi(\overline{a_1},\dots,\overline{a_n},\overline{u})$
\item для всех $a_i \in \mathbb{N}_0$ выполнено $\vdash (\exists x.\varphi(\overline{a_1},\dots,\overline{a_n},x)) \with (\forall p.\forall q.\varphi(\overline{a_1},\dots,\overline{a_n},p)\with \varphi(\overline{a_1},\dots,\overline{a_n},q)\rightarrow p=q)$
\end{enumerate}

\subsection{Соответствие рекурсивных и представимых функций}

\thmm{Теорема.} Любая рекурсивная функция представима в Ф.А.

\thmm{Теорема.} Любая представимая в Ф.А. функция рекурсивна.

\thmm{Теорема.}
Примитивы $Z$, $N$ и $U^k_n$ представимы в Ф.А.


\textbf{Доказательство:}
\begin{itemize}
\item $\zeta(x_1,x_2) := x_2=0$, формальнее: $\zeta(x_1,x_2) := x_1=x_1 \with x_2=0$
\item $\nu(x_1,x_2) := x_2=x_1'$
\item $\upsilon(x_1,\dots,x_n,x_{n+1}) := x_k = x_{n+1}$ 

формальнее: $\upsilon(x_1,\dots,x_n,x_{n+1}) := (\underset{i\ne k,n+1}{\with} x_i=x_i) \with x_k = x_{n+1}$
\end{itemize}

\subsection{Примитив S представим в Ф.А.}
$$S\langle f,g_1,\dots,g_k\rangle(x_1,\dots,x_n) = f(g_1(x_1,\dots,x_n),\dots,g_k(x_1,\dots,x_n))$$
\thmm{Теорема.}

Пусть функции $f,g_1,\dots,g_k$ представимы в Ф.А. Тогда $S\langle f,g_1,\dots,g_k \rangle$ представима в Ф.А.


\textbf{Доказательство:}

Пусть $f$, $g_1$, ..., $g_k$ представляются формулами $\varphi$, $\gamma_1$, ..., $\gamma_k$.

Тогда $S\langle f,g_1,\dots,g_k\rangle$ будет представлена формулой
$$\exists g_1.\dots.\exists g_k.\varphi(g_1,\dots,g_k,x_{n+1})\with\gamma_1(x_1,\dots,x_n,g_1)\with\dots\with\gamma_k(x_1,\dots,x_n,g_k)$$

\subsection{$\beta$-функция Гёделя}

Задача: закодировать последовательность натуральных чисел произвольной длины.

\deff{def:} $\beta$-функция Гёделя: $\mathcal{\beta}(b,c,i) := b \% (1 + (i+1) \cdot c)$\\
Здесь (\%) --- остаток от деления.


\thmm{Теорема} $\beta$-функция Гёделя представима в Ф.А. формулой
$$\hat{\beta}(b,c,i,d) := \exists q.(b = q \cdot (1 + c \cdot (i+1)) + d) \& (d < 1 + c \cdot (i+1))$$

Деление $b$ на $x$ с остатком: найдутся частное $(q)$ и остаток $(d)$, что $b = q\cdot x + d$ и $0 \le d < x$.

\thmm{Теорема} Если $a_0, \dots, a_n \in \mathbb{N}_0$, то найдутся такие $b,c \in \mathbb{N}_0$, что $a_i = \beta(b,c,i)$




\thmm{Теорема:}Китайская теорема об остатках (вариант формулировки): если $u_0, \dots, u_n$ --- попарно взаимно просты, и $0 \le a_i < u_i$, то существует такой $b$, что $a_i = b \% u_i$.


\textbf{Доказательство:}

Положим $c = \max(a_0,\dots,a_n,n)!$ и $u_i = 1+c\cdot(i+1)$.

\begin{itemize}
\item $\text{НОД}(u_i,u_j) = 1$, если $i \ne j$.

Пусть $p$ --- простое, $u_i : p$ и $u_j: p$ ($i < j$). Заметим, что $u_j-u_i = c \cdot (j-i)$. Значит, $c: p$ или $(j-i): p$. Так как $j-i \le n$, то $c : (j-i)$, потому если и $(j-i): p$, всё равно $c : p$. Но и $(1+c\cdot(i+1)): p$, отсюда $1 : p$ --- что невозможно.
\item $0 \le a_i < u_i$.
\end{itemize}

Условия китайской теоремы об остатках выполнены и найдётся $b$, что 
$$a_i = b \% (1 + c\cdot(i+1)) = \beta(b,c,i)$$

\hfill Q.E.D.

\subsection{Примитив <<примитивная рекурсия>> представим в Ф.А.}

Пусть $f: \mathbb{N}^n_0\to\mathbb{N}_0$ и $g:\mathbb{N}^{n+2}_0\to\mathbb{N}_0$ представлены формулами $\varphi$ и $\gamma$. 

Зафиксируем $x_1, \dots, x_n, y \in \mathbb{N}_0$.

\begin{tabular}{lll}
Шаг вычисления & Об. & Утверждение в Ф.А.\\
$R\langle f,g\rangle (x_1,\dots,x_n,0) = f(x_1,\dots,x_n)$ & $a_0$ & $\vdash \varphi(\overline{x_1},\dots,\overline{x_n},\overline{a_0})$\\ 
$R\langle f,g\rangle (x_1,\dots,x_n,1) = g(x_1,\dots,x_n,0,a_0)$ & $a_1$ & $\vdash \gamma(\overline{x_1},\dots,\overline{x_n},0,\overline{a_0},\overline{a_1})$\\ 
$\dots$\\
$R\langle f,g\rangle (x_1,\dots,x_n,y) = g(x_1,\dots,x_n,y-1,a_{y-1})$ & $a_y$ & $\vdash \gamma(\overline{x_1},\dots,\overline{x_n},\overline{y-1},\overline{a_{y-1}},\overline{a_y})$
\end{tabular}

По свойству $\beta$-функции, найдутся $b$ и $c$, что $\beta (b,c,i) = a_i$ для $0 \le i \le y$.


\thmm{Теорема.}

Примитив $R\langle f,g\rangle$ представим в Ф.А. формулой $\rho(x_1,\dots,x_n,y,a)$:
$$\begin{array}{l}\exists b. \exists c. (\exists a_0. \hat{\beta} (b,c,0,a_0) \& \varphi (x_1,...x_n, a_0)) \\
       \&\;\;\;\;\forall k.k < y \rightarrow \exists d . \exists e . \hat{\beta} (b,c,k,d) \& \hat{\beta} (b,c,k',e) \& \gamma (x_1,..x_n,k,d,e) \\
       \&\;\;\;\;\hat{\beta} (b,c,y,a) 
\end{array}$$

\subsection{Представимость рекурсивных функций в Ф.А.}

\thmm{Теорема.} 

Пусть функция $f:\mathbb{N}^{n+1}_0 \to \mathbb{N}_0$ представима в Ф.А. формулой $\varphi(x_1,\dots,x_{n},y,r)$. Тогда примитив $M\langle f\rangle$ представим в Ф.А. формулой $$\mu(x_1,\dots,x_n,y) := \varphi(x_1,\dots,x_n,y,0) \with \forall u.u < y \to \neg\varphi(x_1,\dots,x_n,u,0)$$

\thmm{Теорема.}

Если $f$ --- рекурсивная функция, то она представима в Ф.А.


Индукция по структуре $f$.

\subsection{Рекурсивность представимых в Ф.А. функций}

Фиксируем $f$ и $x_1, x_2, \dots, x_n$. Обозначим $y = f(x_1,x_2,\dots,x_n)$. По представимости нам известна $\varphi$, что $\vdash \varphi(\overline{x_1},\overline{x_2},\dots,\overline{x_n},\overline{y})$. Давайте просто переберём все результаты и доказательства!

\begin{enumerate}
\item Закодируем доказательства натуральными числами.
\item Напишем рекурсивную функцию, проверяющую доказательства на корректность.
\item Параллельный перебор значений и доказательств: $s = 2^y \cdot 3^p$. Переберём все $s$, по $s$ получим $y$ и $p$. Проверим, что $p$ --- код доказательства $\vdash \varphi(\overline{x_1},\overline{x_2},\dots,\overline{x_n},\overline{y})$.
\end{enumerate}

\subsection{Гёделева нумерация}

\begin{enumerate}
\item Отдельный символ.

\begin{tabular}{lc|lc||cll}
Номер & Символ & Номер & Символ & Имя & $k,n$ & Гёделев номер\\
3 & ( &               17 & $\&$ &  0 & $0,0$ & $27 + 6$\\
5 & ) &               19 & $\forall$ & $(')$ & $0,1$ & $27 + 6 \cdot 3$\\
7 & , &               21 & $\exists$ & $(+)$ & $0,2$ & $27 + 6 \cdot 9$\\
9 & . &               23 & $\vdash$ & $(\cdot)$ & $1,2$ & $27 + 6 \cdot 2 \cdot 9$\\
11 & $\neg$ &         $25 + 6\cdot k$ & $x_k$ & $(=)$ & $0,2$ & $29 + 6 \cdot 9$\\
13 & $\rightarrow$ &  $27 + 6\cdot 2^k \cdot 3^n$ & $f_k^n$\\
15 & $\vee$ &         $29 + 6\cdot 2^k \cdot 3^n$ & $P_k^n$ 
\end{tabular}

\item Формула. $\phi \equiv s_0s_1\dots s_{n-1}$. Гёделев номер: $\ulcorner\phi\urcorner = 2^{\ulcorner s_0\urcorner}\cdot 3^{\ulcorner s_1\urcorner} \cdot \dots \cdot p_{n-1}^{\ulcorner s_{n-1}\urcorner}$.

\item Доказательство. $\Pi = \delta_0\delta_1\dots\delta_{k-1}$, его гёделев номер: $\ulcorner\Pi\urcorner = 2^{\ulcorner \delta_0\urcorner}\cdot 3^{\ulcorner \delta_1\urcorner} \cdot \dots \cdot p_{k-1}^{\ulcorner \delta_{k-1}\urcorner}$
\end{enumerate}

\thmm{Теорема.} 

Следующая функция рекурсивна:
$$\text{proof}(f,x_1,x_2,\dots,x_n,y,p) = \left\{\begin{array}{ll} 
1, & \mbox{если} \vdash\phi(\overline{x_1},\overline{x_2},\dots,\overline{x_n},\overline{y}),\\
   & p \mbox{ --- гёделев номер вывода}, f = \ulcorner\phi\urcorner \\ 
0, & \mbox{иначе}
\end{array}\right.$$

Идея доказательства
\begin{enumerate}
\item Проверка доказательства вычислима.
\item Согласно тезису Чёрча, любая вычислимая функция вычислима с помощью рекурсивных функций.
\end{enumerate}

Перебор доказательств

\thmm{Лемма.}

Следующие функции рекурсивны:
\begin{enumerate}
\item Функции $\text{plog}_k(n) = \max\{p: n : k^p\}$, $\text{fst}(x) = \text{plog}_2(x)$ и $\text{snd}(x) = \text{plog}_3(x)$.
\item Числовые литералы: $\overline{k}: \mathbb{N}_0 \rightarrow \mathbb{N}_0$, $\overline{k}(x) = k$.
\end{enumerate}


\thmm{Теорема.}

Если $f: \mathbb{N}^n_0\rightarrow\mathbb{N}_0$, и $f$ представима в Ф.А. формулой $\varphi$, то $f$ --- рекурсивна.

\textbf{Доказательство:}

Пусть заданы $x_1,x_2,\dots,x_n$. Ищем $\langle y, p\rangle$, что $\text{proof}(\ulcorner\varphi\urcorner,x_1,x_2,\dots,x_n,y,p)=1$, напомним: $y = f(x_1,x_2,\dots,x_n)$, $p = \ulcorner\Pi\urcorner$, $\Pi$ --- доказательство $\varphi(\overline{x_1},\overline{x_2},\dots,\overline{x_n},\overline{y})$. 
$$f = S \langle \text{fst}, M\langle S \langle \text{proof}, \overline{\ulcorner\varphi\urcorner}, U^1_{n+1}, U^2_{n+1}, \dots, U^n_{n+1}, S \langle \text{fst},U^{n+1}_{n+1}\rangle, S\langle \text{snd}, U^{n+1}_{n+1}\rangle \rangle \rangle \rangle$$


