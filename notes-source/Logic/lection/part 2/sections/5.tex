\section{Лекция 5.}

\subsection{Введение в исчисление предикатов}



\deff{def:} \textbf{Силлогизм} --- «подытоживание, подсчёт, умозаключение»

\deff{def:} \textbf{Категорический} --- потому, что речь идёт о категориях (в философском смысле).

Определяем некоторые стандартные мыслительные блоки, с которыми у образованной аудитории есть навык работы.
Цель --- сделать неформальный человеческий язык чуть более формальным. Где важно: научный трактат, диспут,
для исключения ошибок в рассуждениях. 

Язык рассуждений понимается единым, без разделения на язык исследователя и предметный.

Пример категорического силлогизма:
$$\infer{\text{Сократ смертен}}{\text{Каждый человек смертен}\quad\quad\text{Сократ есть человек}}$$





Категорический силллогизм соединяет три термина:
\begin{center}
    

\begin{tabular}{l}
предикат (больший термин, P)\\
субъект (меньший термин, S)\\
средний термин (M). 
\end{tabular}
\end{center}
На основании соотношений P и M, а также M и S строим соотношение P и S.

\vspace{0.3cm}
Возможные соотношения:

\begin{tabular}{lll}
A & Affirmato (общеутвердительное) & Матан есть раздел математики (SaP)\\
I & affIrmato (частноутвердительное) & Некоторые разделы математики сложны (SiP)\\
E & nEgo (общеотрицательное) & Никакой человек не знает всю математику\\
O & negO (частноотрицательное) & Некоторые разделы математики --- не матан
\end{tabular}

\deff{def:} Каждому силогизму соответствует \textbf{фигура}

\begin{tabular}{lcccc}
& Фигура 1 & Фигура 2 & Фигура 3 & Фигура 4\\
& 
\tikz{
    \node at (0,1) (M1) { \tiny $M$ };
    \node at (1,1) (P)  { \tiny $P$ };
    \node at (0,0) (S)  { \tiny $S$ };
    \node at (1,0) (M2) { \tiny $M$ };
    \draw (0.85,0.85) -- (0.15,0.85) -- (0.85,0.15) -- (0.15,0.15);
}
&
\tikz{
    \node at (1,1) (M1) { \tiny $M$ };
    \node at (0,1) (P)  { \tiny $P$ };
    \node at (0,0) (S)  { \tiny $S$ };
    \node at (1,0) (M2) { \tiny $M$ };
    \draw (0.15,0.85) -- (0.85,0.85) -- (0.85,0.15) -- (0.15,0.15);
}
&
\tikz{
    \node at (0,1) (M1) { \tiny $M$ };
    \node at (1,1) (P)  { \tiny $P$ };
    \node at (1,0) (S)  { \tiny $S$ };
    \node at (0,0) (M2) { \tiny $M$ };
    \draw (0.85,0.85) -- (0.15,0.85) -- (0.15,0.15) -- (0.85,0.15);
}
&
\tikz{
    \node at (1,1) (M1) { \tiny $M$ };
    \node at (0,1) (P)  { \tiny $P$ };
    \node at (1,0) (S)  { \tiny $S$ };
    \node at (0,0) (M2) { \tiny $M$ };
    \draw (0.15,0.85) -- (0.85,0.85) -- (0.15,0.15) -- (0.85,0.15);
}

\\

Большая посылка: & M—P & P—M & M—P & P—M\\
Меньшая посылка: & S—M & S—M & M—S & M—S\\
Заключение: & S—P & S—P & S—P &S—P 
\end{tabular}

\vspace{0.3cm}
Расстановка соотношений вместо <<—>> в фигуре --- модус. Например, тут --- фигура 1, aaa.

$$\infer{\text{Сократ смертен}}{\text{Каждый человек смертен}\quad\quad\text{Сократ есть человек}}$$

Как этим пользоваться: по умозаключению (на русском языке) определяем, где в нём P, M, S и каковы
между ними соотношения, находим соответствующую фигуру и модус, а дальше
определяем силлогизм и его свойства в соответствии со следующими правилами.

Не все модусы осмысленны, большинство некорректно. Например фигура 1, aae:
$$\infer{\text{Сократ не есть смертен}}{\text{Каждый человек смертен}\quad\quad\text{Сократ есть человек}}$$

Список всех правильных модусов (из них выделяют \emph{слабые}, выводящие частное соотношение при возможности общего --- указаны курсивом):

{\small
\begin{center}\begin{tabular}{llll}
Фигура 1 &Фигура 2 &Фигура 3 &Фигура 4\\
Barbara &Cesare &{\color{gray}Darapti} &\color{gray}Bramantip\\
Celarent &Camestres &Disamis &Camenes\\
Darii &Festino &Datisi &Dimaris\\
Ferio &Baroco &{\color{gray}Felapton} &{\color{gray}Fesapo}\\
\it Barbari &\it Cesaro &Bocardo &Fresison\\
\it Celaront &\it Camestros &Ferison &\it Camenos
\end{tabular}\end{center}}

Некоторые модусы требуют непустоты M: это все слабые модусы и четыре сильных (указаны серым), например Darapti:
$$\infer{\text{Некоторые лошади имеют рог}}{\text{Все единороги имеют рог}\quad\quad\text{Все единороги суть лошади}}$$

\deff{Ограничения языка исчисления высказываний:}

$$\infer{\text{Сократ смертен}}{\text{{\color{blue}Каждый} человек смертен}\quad\quad\text{Сократ {\color{red}есть} человек}}$$

\begin{center}Цель: увеличить формализованную часть метаязыка.\end{center}

Мы неформально знакомы с {\color{red}предикатами} ($P: D \rightarrow V$) и {\color{blue}кванторами} ($\forall x.H(x) \rightarrow S(x)$).
$$
\infer{{\color{red}S}(\text{Сократ})}{{\color{blue}\forall} x.{\color{red}H}(x)\rightarrow {\color{red}S}(x)\quad\quad {\color{red}H}(\text{Cократ})}
$$

\subsection{Язык исчисления предикатов}

\textbf{Пример:}
$$\color{blue}\forall {\color{red}x}.{\color{red}\sin x} = {\color{red}0} \vee {\color{red}(\sin x)^2+1} > {\color{red}1}$$
\begin{enumerate}
\item Предметные (здесь: числовые) выражения
\begin{enumerate}
\item Предметные переменные ($\color{red}x$).
\item Одно- и двухместные функциональные символы <<синус>>, <<возведение в квадрат>> и <<сложение>>.
\item Нульместные функциональные символы <<ноль>> ($\color{red}0$) и <<один>> ($\color{red}1$).
\end{enumerate}
\item Логические выражения
\begin{enumerate}
\item Предикатные символы <<равно>> и <<больше>>
\end{enumerate}
\end{enumerate}

\subsubsection{Формальное определение}

\begin{enumerate}
\item Два типа: предметные и логические выражения.
\item Предметные выражения: метапеременная {\color{blue}$\theta$}.
\begin{itemize}
\item Предметные переменные: {\color{blue}$a$}, {\color{blue}$b$}, {\color{blue}$c$}, \dots, метапеременные {\color{blue}$x$}, {\color{blue}$y$}.
\item Функциональные выражения: {\color{blue}$f(\theta_1,\dots,\theta_n)$}, метапеременные {\color{blue}$f$}, {\color{blue}$g$}, \dots
\item Примеры: {\color{blue}$r$}, {\color{blue}$q(p(x,s),r)$}.
\end{itemize}
\item Логические выражения: метапеременные {\color{blue}$\alpha$}, {\color{blue}$\beta$}, {\color{blue}$\gamma$}, \dots
\begin{itemize}
\item Предикатные выражения: {\color{blue}$P(\theta_1,\dots,\theta_n)$}, метапеременная {\color{blue}$P$}.
Имена: {\color{blue}$A$}, {\color{blue}$B$}, {\color{blue}$C$}, \dots
\item Связки: {\color{blue}$(\varphi\vee\psi)$}, {\color{blue}$(\varphi\with\psi)$}, {\color{blue}$(\varphi\rightarrow\psi)$}, 
   {\color{blue}$(\neg\varphi)$}.
\item Кванторы: {\color{blue}$(\forall x.\varphi)$} и {\color{blue}$(\exists x.\varphi)$}.
\end{itemize}
\end{enumerate}


\deff{Сокращение записи и метаязык:}

\begin{enumerate}
\item Метапеременные:
\begin{itemize}
\item $\color{blue}\psi$, $\color{blue}\phi$, $\color{blue}\pi$, \dots --- формулы
\item $\color{blue}P$, $\color{blue}Q$, \dots --- предикатные символы
\item $\color{blue}\theta$, \dots --- термы
\item $\color{blue}f$, $\color{blue}g$, \dots --- функциональные символы
\item $\color{blue}x$, $\color{blue}y$, \dots --- предметные переменные
\end{itemize}

\item Скобки --- как в И.В.; квантор --- жадный:
\begin{center}${\color{blue}(\forall a.} \underbrace{{\color{blue}A \vee B \vee C \rightarrow \exists b.}
                    \underbrace{\color{blue}D \with \neg E}_{\exists b.\dots}}_{\forall a.\dots} \color{blue}) \with F$\end{center}

\item Дополнительные обозначения при необходимости:
\begin{itemize}
\item $\color{blue}(\theta_1 = \theta_2)$ вместо $\color{blue}E(\theta_1,\theta_2)$
\item $\color{blue}(\theta_1 + \theta_2)$ вместо $\color{blue}p(\theta_1,\theta_2)$
\item $\color{blue}0$ вместо $\color{blue}z$
\item \dots
\end{itemize}
\end{enumerate}

\subsection{Теория моделей исчисления предикатов}

Напомним формулу:
$$\forall x.\sin x = 0 \vee (\sin x)^2+1 > 1$$
Без синтаксического сахара:
$$\forall x.{\color{blue}E}(s (x),z)~\vee~{\color{blue}G} (p(q(s (x)),o), o)$$
$${\color{blue}\forall} x.{\color{blue}E}(s (x),z) {\color{blue}~\vee~} {\color{blue}G} (p(q(s (x)),o), o)$$
$${\color{blue}\forall} {\color{red}x}.{\color{blue}E}(s ({\color{red}x}),z) {\color{blue}~\vee~} {\color{blue}G} 
   (p(q(s ({\color{red}x})),o),o)$$
$${\color{blue}\forall} {\color{red}x}.{\color{blue}E}({\color{red}s} ({\color{red}x}),{\color{red}z}) {\color{blue}~\vee~} {\color{blue}G} 
   ({\color{red}p}({\color{red}q}({\color{red}s} ({\color{red}x})),{\color{red}o}), {\color{red}o})$$

\begin{enumerate}
\item {\color{blue}Истинностные (логические) значения: }
\begin{enumerate}
\item предикаты (в том числе пропозициональные переменные = нульместные предикаты);
\item логические связки и кванторы.
\end{enumerate}

\item {\color{red}Предметные значения:}
\begin{enumerate}
\item предметные переменные;
\item функциональные символы (в том числе константы = нульместные функциональные символы)
\end{enumerate}
\end{enumerate}

\subsubsection{Оценка исчисления предикатов}

\deff{def:} \textbf{Оценка} --- упорядоченная четвёрка $\langle D, F, P, E \rangle$, где:

\begin{enumerate}
\item $D \neq \varnothing$ --- предметное множество;
\item $F$ --- оценка для функциональных символов; пусть $f_n$ --- $n$-местный функциональный символ:
 $$F_{f_n}: D^n \rightarrow D$$

\item $P$ --- оценка для предикатных символов; пусть $T_n$ --- $n$-местный предикатный символ:
 $$P_{T_n}: D^n \rightarrow V\quad\quad\quad V = \{\text{И}, \text{Л}\}$$

\item $E$ --- оценка для предметных переменных.
 $$E(x) \in D$$
\end{enumerate}

Запись и сокращения записи подобны исчислению высказываний: $$\llbracket \phi \rrbracket \in V,\quad
      \llbracket Q(x,f(x))\vee R\rrbracket^{x := 1, f(t) := t^2, R := \text{И}} = \text{И}$$

\begin{enumerate}
\item Правила для связок $\vee$, $\with$, $\neg$, $\rightarrow$ остаются прежние;
\item $$\llbracket f_n (\theta_1, \theta_2, \dots, \theta_n) \rrbracket = F_{f_n} (\llbracket\theta_1\rrbracket,
          \llbracket\theta_2\rrbracket, \dots, \llbracket\theta_n\rrbracket)$$
\item $$\llbracket P_n (\theta_1, \theta_2, \dots, \theta_n) \rrbracket = P_{T_n} (\llbracket\theta_1\rrbracket,
          \llbracket\theta_2\rrbracket, \dots, \llbracket\theta_n\rrbracket)$$
\item $$\llbracket \forall x.\phi \rrbracket = \left\{\begin{array}{ll}
   \text{И}, & \text{если } \llbracket\phi\rrbracket^{x := t} = \text{И}\text{ при всех } t \in D\\
   \text{Л}, & \text{если найдётся } t \in D, \text{ что } \llbracket\phi\rrbracket^{x := t} = \text{Л}
  \end{array}\right.$$
\item $$\llbracket \exists x.\phi \rrbracket = \left\{\begin{array}{ll}
   \text{И}, & \text{если найдётся } t \in D, \text{ что } \llbracket\phi\rrbracket^{x := t} = \text{И}\\
   \text{Л}, & \text{если } \llbracket\phi\rrbracket^{x := t} = \text{Л}\text{ при всех } t \in D
  \end{array}\right.$$
\end{enumerate}


\textbf{Пример:} Оценим:
$$\llbracket \forall a.\exists b.\neg a+1 = b \rrbracket$$

\subsubsection{Общезначимость и свободные переменные}


\deff{def:} Формула исчисления предикатов \textbf{общезначима}, если истинна при любой оценке:
$$\models\phi$$

То есть истинна при любых $D$, $F$, $P$ и $E$.

\textbf{Пример:}

$$\llbracket\forall x.Q(f(x))\vee\neg Q(f(x))\rrbracket = \text{И}$$

\textbf{Доказательство:}

Фиксируем $D$, $F$, $P$, $E$. Пусть $x \in D$.
Обозначим $P_{Q}(F_{f}(E_x))$ за $t$.
Ясно, что $t \in V$. Разберём случаи.
\begin{itemize}
\item Если $t = \text{И}$, то $\llbracket Q(f(x))\rrbracket^{Q(f(x)):=t} = \text{И}$,
  потому $\llbracket Q(f(x))\vee\neg Q(f(x))\rrbracket^{Q(f(x)):=t} = \text{И}$
\item Если $t = \text{Л}$, то $\llbracket \neg Q(f(x))\rrbracket^{Q(f(x)):=t} = \text{И}$, потому
  всё равно $\llbracket Q(f(x))\vee\neg Q(f(x))\rrbracket^{Q(f(x)):=t} = \text{И}$
\end{itemize}


\deff{def:} \textbf{Вхождение подформулы} в формулу --- это позиция первого символа этой подформулы в формуле.
$$\text{Вхождения }{\color{blue}x}\text{ в формулу:}\quad (\forall {\color{blue}x}.A({\color{blue}x}) \vee \exists {\color{blue}x}.B({\color{blue}x})) \vee C({\color{blue}x})$$

\deff{def:} Рассмотрим формулу $\forall x.\psi$ (или $\exists x.\psi$). Здесь переменная $x$ \textbf{связана} в $\psi$.
Все вхождения переменной $x$ в $\psi$ --- связанные.

\deff{def:} Вхождение $x$ в $\psi$ \textbf{свободное}, если не находится в области действия никакого квантора по $x$.
Переменная входит свободно в $\psi$, если имеет хотя бы одно свободное вхождение. $FV(\psi), FV(\Gamma)$ --- множества свободных
переменных в $\psi$, в $\Gamma$

\textbf{Пример:}

$$\psi[x := \theta] := \left\{\begin{array}{ll}\psi, & \psi\equiv y, y \not\equiv x\\
                                  \psi, & \psi\equiv\forall x.\pi \textrm{ или } \psi\equiv\exists x.\pi\\
                                  \pi[x := \theta] \star \rho [x := \theta], & \psi\equiv \pi\star\rho\\
                                  \theta, & \psi\equiv x\\
                                  \forall y.\pi[x := \theta], & \psi \equiv \forall y.\pi \textrm{ и } y \not\equiv x\\
                                  \exists y.\pi[x := \theta], & \psi \equiv \exists y.\pi \textrm{ и } y \not\equiv x
\end{array}\right.$$

\deff{def:} Терм $\theta$ \deff{свободен для подстановки вместо $x$} в $\psi$ ($\psi[x := \theta]$), если 
ни одно свободное вхождение переменных в $\theta$ не станет связанным после подстановки.

\begin{center}\begin{tabular}{c|c}
Свобода есть & Свободы нет\\\hline
$(\forall x.P(y)) [y := z]$ & $(\forall x.P(y)) [y := x]$\\
$(\forall y.\forall x.P(x)) [x := y]$ & $(\forall y.\forall x.P(t)) [t := y]$
\end{tabular}\end{center}

\subsection{Теория доказательств исчисления предикатов}

Рассмотрим язык исчисления предикатов. Возьмём все схемы аксиом классического исчисления высказываний и добавим ещё две схемы аксиом 
(здесь везде $\theta$ свободен для подстановки вместо $x$ в $\varphi$):

\begin{tabular}{ll}
11. & $(\forall x.\varphi) \rightarrow \varphi[x:=\theta]$\\
12. & $\varphi[x:=\theta] \rightarrow \exists x.\varphi$ 
\end{tabular}

Добавим ещё два правила вывода (здесь везде $x$ не входит свободно в $\varphi$):
$$\infer[\text{Правило для }\forall]{\varphi\rightarrow\forall x.\psi}{\varphi\rightarrow\psi}$$
$$\infer[\text{Правило для }\exists]{(\exists x.\psi)\rightarrow\varphi}{\psi\rightarrow\varphi}$$

\deff{def:} Доказуемость, выводимость, полнота, корректность --- аналогично исчислению высказываний.

\subsection{Теоремы о исчислении предикатов}

\subsubsection{Теорема о дедукции}

Если $\Gamma\vdash\alpha\rightarrow\beta$, то $\Gamma,\alpha\vdash\beta$.
Если $\Gamma,\alpha\vdash\beta$ и в доказательстве не применяются правила для кванторов 
по свободным переменным из $\alpha$, то $\Gamma\vdash\alpha\rightarrow\beta$.

\textbf{Доказательство:}

$(\Rightarrow)$ --- как в КИВ $(\Leftarrow)$ --- та же схема, два новых случая.

Перестроим: $\delta_1, \delta_2, \dots, \delta_n \equiv \beta$ в $\alpha\rightarrow\delta_1, \alpha\rightarrow\delta_2, \dots, \alpha\rightarrow\delta_n$.

Дополним: обоснуем $\alpha\rightarrow\delta_n$, если предыдущие уже обоснованы.

Два новых похожих случая: правила для $\forall$ и $\exists$. Рассмотрим $\forall$.

Доказываем $(n)\ \ \alpha\rightarrow\psi\rightarrow\forall x.\varphi$ (правило для $\forall$), значит, доказано  
$(k)\ \ \alpha\rightarrow\psi\rightarrow\varphi$.

\begin{tabular}{lll}
$(n-0.9) $ & $(\alpha\rightarrow\psi\rightarrow\varphi)\rightarrow(\alpha\with\psi)\rightarrow\varphi$ & Т. о полноте КИВ\\
$(n-0.6)$ & $(\alpha\with\psi)\rightarrow\varphi$ & M.P. $k$,$n-0.8$\\
$(n-0.4)$ & $(\alpha\with\psi)\rightarrow\forall x.\varphi$ & Правило для $\forall$, $n-0.6$\\
$(n-0.3)$ & $((\alpha\with\psi)\rightarrow\forall x.\varphi)\rightarrow(\alpha\rightarrow\psi\rightarrow\forall x.\varphi)$ & Т. о полноте КИВ\\
$(n)$ & $\alpha\rightarrow\psi\rightarrow\forall x.\varphi$ & M.P. $n-0.4$, $n-0.2$
\end{tabular}

\hfill Q.E.D.


\deff{def:} $\gamma_1,\gamma_2,\dots,\gamma_n\models\alpha$, если $\alpha$ выполнено всегда, когда выполнено $\gamma_1,\gamma_2,\dots,\gamma_n$.


\thmm{Теорема}

Если $\Gamma\vdash\alpha$ и в доказательстве не используются кванторы по свободным
переменным из $\Gamma$, то $\Gamma\models\alpha$


\subsubsection{Корректность подстановки}

\thmm{Теорема.}

Если $\theta$ свободен для подстановки 
вместо $x$ в $\varphi$, то $\llbracket\varphi\rrbracket^{x := \llbracket\theta\rrbracket} = \llbracket\varphi[x := \theta]\rrbracket$

\textbf{Доказательство (индукция по структуре $\varphi$)}

\begin{itemize}
\item База: $\varphi$ не имеет кванторов. Очевидно.
\item Переход: пусть справедливо для $\psi$. Покажем для $\varphi = \forall y.\psi$. 
\begin{itemize}
\item $x=y$ либо $x \notin FV(\psi)$. Тогда: 
$\llbracket\forall y.\psi\rrbracket^{x := \llbracket\theta\rrbracket} = \llbracket\forall y.\psi\rrbracket = \llbracket(\forall y.\psi)[x := \theta]\rrbracket$

\item $x \ne y$. Тогда: $\llbracket\forall y.\psi\rrbracket^{x := \llbracket\theta\rrbracket} = 
  \llbracket\psi\rrbracket^{y \in D; x := \llbracket\theta\rrbracket} = \dots$

{\color{olive}Свобода для подстановки: $y\notin\theta$.}
 $$\dots = \llbracket\psi\rrbracket^{x := \llbracket\theta\rrbracket; y \in D} = \dots$$

{\color{olive}Индукционное предположение.}

 $$\dots = \llbracket\psi[x := \theta]\rrbracket^{y \in D} = 
\llbracket\forall y.(\psi[x := \theta])\rrbracket = \dots$$

{\color{olive}Но $\forall y.(\psi[x := \theta]) \equiv (\forall y.\psi) [x := \theta]$ (как текст). Отсюда:}

$$\dots = \llbracket(\forall y.\psi)[ x := \theta]\rrbracket$$
\end{itemize}
\end{itemize}

\subsubsection{Корректность исчисления предикатов}

\thmm{Теорема.} Если $\Gamma \vdash \alpha$ и в доказательстве не используются кванторы по свободным переменным из $FV(\Gamma)$, то $\Gamma \models \alpha$


\textbf{Доказательство:}

Фиксируем $D, F, P$. Индукция по длине доказательства $\alpha$: при любом $E$ выполнено $\Gamma\models\alpha$ 
при длине доказательства $n$, покажем для $n+1$. 
\begin{itemize}
\item Схемы аксиом (1)..(10), правило M.P.: аналогично И.В.
\item Схемы (11) и (12), например, схема $(\forall x.\varphi) \rightarrow \varphi [x := \theta]$:

$$\llbracket (\forall x.\varphi) \rightarrow \varphi [x := \theta]\rrbracket = \llbracket ((\forall x.\varphi) \rightarrow \varphi) [x := \theta] \rrbracket =
  \llbracket (\forall x.\varphi) \rightarrow \varphi \rrbracket ^ { x := \llbracket\theta\rrbracket } = \text{И}$$

\item Правила для кванторов: например, введение $\forall$:

Пусть $\llbracket \psi \rightarrow \varphi \rrbracket = \text{И}$. Причём $x \notin FV(\Gamma)$ и $x \notin FV(\psi)$. То есть,
при любом $\mathcal{x}$ выполнено $\llbracket \psi \rightarrow \varphi \rrbracket^{x := \mathcal{x}} = \text{И}$. Тогда
$\llbracket \psi \rightarrow (\forall x.\varphi) \rrbracket = \text{И}$.
\end{itemize}

