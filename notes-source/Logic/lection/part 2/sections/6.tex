\section{Лекция 6.}

\subsection{Теорема о полноте}

\textbf{Общая идея доказательства:}

\begin{enumerate}
\item Надо справиться со слишком большим количеством вариантов.
      Модель задаётся как $\langle D,F,P,E \rangle$.
\item Для оценки в модели важно только какие формулы истинны. Поэтому факторизуем модели по истинности формул:
      модели $\mathcal{M}_1$ и $\mathcal{M}_2$ <<похожи>>, если
      $\llbracket \varphi \rrbracket_{\mathcal{M}_1} = \llbracket \varphi \rrbracket_{\mathcal{M}_2}$
      при всех $\varphi$.
\item Поступим так:
    \begin{enumerate}
       \item построим эталонное множество моделей $\mathfrak{M}$, каждая модель из него соответствует какому-то своему классу эквивалентности моделей;
       \item докажем полноту $\mathfrak{M}$: если каждая $\mathcal{M} \in \mathfrak{M}$ предполагает $\mathcal{M}\models\varphi$,
             то $\vdash\varphi$;
       \item заметим, что если $\models\varphi$, то каждая $\mathcal{M} \in \mathfrak{M}$ предполагает $\mathcal{M}\models\varphi$.
    \end{enumerate}
\item В ходе доказательства нас ждёт множество технических препятствий.
\end{enumerate}


\deff{def:} $\Gamma$ --- \textbf{непротиворечивое множество формул},
если $\Gamma\not\vdash\alpha\with\neg\alpha$ для любого $\alpha$

Примеры:
\begin{itemize}
\item непротиворечиво: 
\begin{itemize}
\item $\Gamma = \{A \rightarrow B \rightarrow A\}$
\item $\Gamma = \{P(x,y)\rightarrow\neg P(x,y), \forall x.\forall y.\neg P(x,y)\}$;
\end{itemize}
\item противоречиво: 
\begin{itemize}
\item $\Gamma = \{P\rightarrow\neg P, \neg P \rightarrow P\}$

так как
$P\rightarrow\neg P, \neg P \rightarrow P \ \vdash\  \neg P \with \neg\neg P$
\end{itemize}
\item и ещё непротиворечиво: $\Gamma = \{P(1), P(2), P(3), \dots\}$
\end{itemize}



\deff{def:} $\Gamma$ --- \textbf{полное непротиворечивое множество замкнутых бескванторных формул},
если:
\begin{enumerate}
\item $\Gamma$ содержит только замкнутые бескванторные формулы;
\item если $\alpha$ --- некоторая замкнутая бескванторная формула, то либо $\alpha\in\Gamma$, либо $\neg\alpha\in\Gamma$.
\end{enumerate}


\deff{def:} $\Gamma$ --- \textbf{полное непротиворечивое множество замкнутых формул}, если:
\begin{enumerate}
\item $\Gamma$ содержит только замкнутые формулы;
\item если $\alpha$ --- некоторая замкнутая формула, то либо $\alpha \in \Gamma$, либо $\neg\alpha \in \Gamma$.
\end{enumerate}


\thmm{Теорема:}

Пусть $\Gamma$ --- непротивочивое множество замкнутых (бескванторных) формул. Тогда, какова бы ни была
замкнутая (бескванторная) формула $\varphi$, хотя бы $\Gamma \cup \{\varphi\}$ или $\Gamma \cup \{\neg\varphi\}$ ---
непротиворечиво

\textbf{Доказательство:}

Пусть это не так и найдутся такие $\Gamma$, $\varphi$ и $\alpha$, что
 $$\begin{array}{rl}\Gamma,\varphi & \vdash \alpha\with\neg\alpha\\
                    \Gamma,\neg\varphi & \vdash \alpha \with\neg\alpha\end{array}$$

Тогда по лемме об исключении гипотезы
$$\Gamma\vdash \alpha\with\neg\alpha$$

То есть $\Gamma$ не является непротиворечивым. Противоречие.

\hfill Q.E.D.

\thmm{Теорема.}

Пусть $\Gamma$ --- непротиворечивое множество замкнутых (бескванторных) формул. Тогда
найдётся полное непротиворечивое множество замкнутых (бескванторных) формул $\Delta$, что
$\Gamma \subseteq \Delta$

\textbf{Доказательство:}
\begin{enumerate}
\item Занумеруем все формулы (их счётное количество): $\varphi_1, \varphi_2, \dots$
\item Построим семейство множеств $\{\Gamma_i\}$:

\begin{tabular}{cc}
$\Gamma_0 = \Gamma$  &
$\Gamma_{i+1} = \left\{\begin{array}{ll}\Gamma_i \cup \{\varphi_i\},& \mbox{ если } \Gamma_i \cup \{\varphi_i\} \mbox{ непротиворечиво}\\
                                               \Gamma_i \cup \{\neg\varphi_i\},& \mbox{ иначе}\end{array}\right.$
\end{tabular}
\item Итоговое множество $$\Delta = \bigcup_i \Gamma_i$$
\item Непротиворечивость $\Delta$ не следует из индукции --- индукция гарантирует непротиворечивость
      только $\Gamma_i$ при натуральном (т.е. \emph{конечном}) $i$, потому\dots
\end{enumerate}

\hfill Q.E.D.

{Завершение доказательства теоремы о полноте}



$\Delta$ непротиворечиво:
  \begin{enumerate}
    \item Пусть $\Delta$ противоречиво, то есть $$\Delta \vdash \alpha\with\neg\alpha$$
    \item Доказательство конечной длины и использует конечное количество гипотез $\{\delta_1, \delta_2, \dots, \delta_n\} \subset \Delta$,
          то есть $$\delta_1, \delta_2, \dots, \delta_n \vdash \alpha\with\neg\alpha$$
    \item Пусть $\delta_i \in \Gamma_{d_i}$, тогда $$\Gamma_{d_1}\cup \Gamma_{d_2}\cup \dots\cup \Gamma_{d_n} \vdash \alpha\with\neg\alpha$$
    \item Но $\Gamma_{d_1} \cup \Gamma_{d_2} \cup \dots \cup \Gamma_{d_n} = \Gamma_{\max(d_1,d_2,\dots,d_n)}$,
          которое непротиворечиво, и потому $$\Gamma_{d_1}\cup \Gamma_{d_2}\cup \dots\cup \Gamma_{d_n} \not\vdash \alpha\with\neg\alpha$$
  \end{enumerate}

\hfill Q.E.D

\subsection{Модели для множеств формул}

\deff{def:} \textbf{Моделью для множества формул} $F$ назовём такую модель $\mathcal{M}$, что
    при всяком $\varphi \in F$ выполнено $\llbracket\varphi\rrbracket_\mathcal{M} = \text{И}$.

Альтернативное обозначение: $\mathcal{M}\models\varphi$.

\thmm{Теорема.}

Любое непротиворечивое множество замкнутых бескванторных формул имеет модель.

\subsection{Конструкция модели}

\deff{def:} 
Пусть $M$ --- полное непротиворечивое множество замкнутых бескванторных формул. Тогда
модель $\mathcal{M}$ задаётся так:
\begin{enumerate}
\item $D$ --- множество всевозможных предметных выражений без предметных переменных. Оно непусто (язык обычно содержит много нуль-местных функций), 
но если пусто --- добавим туда какое-нибудь одно значение, пусть ``z''.
\item $\llbracket f(\theta_1,\dots,\theta_n) \rrbracket = \mbox{``f(''} + \llbracket\theta_1\rrbracket + \mbox{ ``,'' }
    + \dots + \mbox{ ``,'' } + \llbracket\theta_n\rrbracket + \mbox {``)'' } $
\item $\llbracket P(\theta_1,\dots,\theta_n)\rrbracket = \left\{
  \begin{array}{ll} \mbox{И}, &\mbox{ если } P(\theta_1,\dots,\theta_n) \in M\\
                   \mbox{Л}, &\mbox{ иначе}\end{array}\right.$
\item Так как $D \ne \varnothing$, то найдётся $z \in D$. Тогда $\llbracket x \rrbracket = z$. Это ничему не помешает, так как формулы замкнуты.
\end{enumerate}


\thmm{Лемма.}

Пусть $\varphi$ --- бескванторная формула, тогда $\mathcal{M}\models\varphi$ тогда и только тогда, когда $\varphi\in M$.

\textbf{Доказательство (индукция по длине формулы $\varphi$)}
\begin{enumerate}
\item База. $\varphi$ --- предикат. Требуемое очевидно по определению $\mathcal{M}$.
\item Переход. Пусть $\varphi = \alpha\star\beta$ (или $\varphi=\neg\alpha$), причём $\mathcal{M}\models\alpha$ ($\mathcal{M}\models\beta$)
   тогда и только тогда, когда $\alpha\in M$ ($\beta\in M$).

Тогда покажем требуемое для каждой связки в отдельности. А именно, для каждой связки покажем два утверждения:
\begin{enumerate}
\item если $\mathcal{M}\models\alpha\star\beta$, то $\alpha\star\beta \in M$.
\item если $\mathcal{M}\not\models\alpha\star\beta$, то $\alpha\star\beta \notin M$.
\end{enumerate}
\end{enumerate}
\hfill Q.E.D.


Если $\varphi = \alpha\to\beta$ и для любой формулы $\zeta$, более короткой, чем $\varphi$, выполнено
$\mathcal{M}\models\zeta$ тогда и только тогда, когда $\zeta\in M$, тогда:
\begin{enumerate}
\item если $\mathcal{M}\models\alpha\to\beta$, то $\alpha\to\beta\in M$;
\item если $\mathcal{M}\not\models\alpha\to\beta$, то $\alpha\to\beta\notin M$.
\end{enumerate}

\textbf{Доказательство (разбором случаев)}
\begin{enumerate}
\item $\mathcal{M}\models\alpha\to\beta$: $\llbracket\alpha\rrbracket = \text{Л}$. 
Тогда по предположению $\alpha\notin M$, потому по полноте
$\neg\alpha\in M$. И, поскольку в ИВ $\neg\alpha\vdash\alpha\to\beta$, то $M \vdash \alpha\to\beta$. 
Значит, $\alpha\to\beta \in M$, иначе по полноте $\neg(\alpha\to\beta) \in M$, что делает $M$ противоречивым.
\item $\mathcal{M}\models\alpha\to\beta$: $\llbracket\alpha\rrbracket = \text{И}$ и $\llbracket\beta\rrbracket = \text{И}$. Рассуждая аналогично,
используя $\alpha,\beta\vdash\alpha\to\beta$, приходим к $\alpha\to\beta \in M$.
\item $\mathcal{M}\not\models\alpha\to\beta$. Тогда $\llbracket\alpha\rrbracket=\text{И}$,
$\llbracket\beta\rrbracket=\text{Л}$, то есть $\alpha\in M$ и $\neg\beta\in M$. 
Также, $\alpha,\neg\beta\vdash\neg(\alpha\to\beta)$, отсюда $M\vdash\neg(\alpha\to\beta)$. 
Предположим, что $\alpha\to\beta\in M$, то $M\vdash\alpha\to\beta$ --- отсюда
$\alpha\to\beta\notin M$.
\end{enumerate}



Завершение доказательства теоремы о существовании модели

\textbf{Доказательство:}

Пусть $M$ --- непротиворечивое множество замкнутых бескванторных формул.

По теореме о пополнении существует $M'$ --- полное непротиворечивое множество замкнутых бескванторных формул,
что $M \subseteq M'$.

По лемме $M'$ имеет модель, эта модель подойдёт для $M$.

\subsection{Теорема Гёделя о полноте}

\thmm{Теорема Гёделя о полноте исчисления предикатов}

Если $M$ --- непротиворечивое множество замкнутых формул, то оно имеет модель.


\textbf{Схема доказательства}
\begin{center}
\tikz{
	\node (M) at (0,3) {$M$};
	\node (M1) at (1,0) {$M^\text{Б}$};
        \node (Md1) at (6,0) {$\mathcal{M}$};
	\node (Md) at (7,3) {$\mathcal{M}$};
        \draw[->] (M) -- node[pos=0.2,right]{\hspace{0.2cm}\begin{minipage}{4cm}сохраняет\\непротиворечивость\end{minipage}} (M1);
	\draw[->] (M1) -- node[below]{\begin{minipage}{4cm}теорема о\\существовании модели\end{minipage}} (Md1);
	\draw[->] (Md1) -- node[pos=0.8,right]{тоже модель} (Md);
	\draw[dashed] (-2,1) -- (9,1);
	\node[above] (Q) at (-2,1) {\it Формулы с кванторами};
	\node[below] (Qf) at (-2,1) {\it Бескванторные};
}
\end{center}


\deff{def:}
Формула $\varphi$ имеет \textbf{поверхностные кванторы (находится в предварённой форме)}, если
соответствует грамматике
$$\varphi ::= \forall x.\varphi\ |\ \exists x.\varphi\ |\ \tau$$
где $\tau$ --- формула без кванторов

\thmm{Теорема.} 

Для любой замкнутой формулы $\psi$ найдётся такая формула $\varphi$ с поверхностными кванторами,
что $\vdash \psi\to\varphi$ и $\vdash\varphi\to\psi$

Индукция по структуре, применение теорем о перемещении кванторов.

Построение $M^*$

\begin{itemize}
\item Пусть $M$ --- полное непротиворечивое множество замкнутых формул с поверхностными кванторами (очевидно, счётное). 
 Построим семейство непротиворечивых множеств замкнутых формул $M_k$.
\item Пусть $d^k_i$ --- семейство \emph{свежих} констант, в $M$ не встречающихся.
\item Индуктивно построим $M_k$:
\begin{itemize}
\item База: $M_0 = M$
\item Переход: положим $M_{k+1} = M_k \cup S$, где множество $S$ получается перебором всех формул $\varphi_i \in M_k$.
\begin{enumerate}
\item $\varphi_i$ --- формула без кванторов, пропустим;
\item $\varphi_i = \forall x.\psi$ --- добавим к $S$ все формулы вида $\psi [x := \theta]$, где
$\theta$ --- всевозможные замкнутые термы, использующие символы из $M_k$;
\item $\varphi_i = \exists x.\psi$ --- добавим к $S$ формулу $\psi [x := d^{k+1}_i]$, где $d^{k+1}_i$ --- некоторая
свежая, ранее не использовавшаяся в $M_k$, константа.
\end{enumerate}
\end{itemize}
\end{itemize}

\thmm{Лемма.} Если $M$ непротиворечиво, то каждое множество из $M_k$ --- непротиворечиво

Доказательство по индукции, база очевидна ($M_0 = M$). 
Переход: 
\begin{itemize}
\item пусть $M_k$ непротиворечиво, но $M_{k+1}$ --- противоречиво: $M_k, M_{k+1}\setminus M_k \vdash A\with\neg A$. 
\item Тогда (т.к. доказательство finite длины):
$M_k, \gamma_1, \gamma_2, \dots,\gamma_n\vdash A\with\neg A$
, где $\gamma_i \in M_{k+1}\setminus M_k$. 
\item По теореме о дедукции: $M_k\vdash \gamma_1\to\gamma_2\to\dots\to\gamma_n\to A\with\neg A$. 
\item Научимся выкидывать первую посылку: $M_k\vdash \gamma_2\to\dots\to\gamma_n\to A\with\neg A$. 
\item И по индукции придём к противоречию: $M_k \vdash A\with\neg A$.
\end{itemize}
\hfill Q.E.D.

\thmm{Лемма.}
Если $M_k\vdash\gamma\to W$ и $\gamma\in M_{k+1}\setminus M_k$, то $M_k\vdash W$.


\textbf{Доказательство:}

Покажем, как дополнить доказательство до $M_k\vdash W$, в зависимости от происхождения $\gamma$:

\begin{itemize}
\item Случай $\forall x.\varphi$: $\gamma = \varphi[x:=\theta]$.

Допишем в конец доказательства:

\begin{tabular}{ll}
$\forall x.\varphi$ & (гипотеза)\\
$(\forall x.\varphi)\to(\varphi[x:=\theta])$ & (сх. акс. 11)\\
$\gamma$  & (M.P.) \\
$W$ & (M.P.)
\end{tabular}

\item Случай $\exists x.\varphi$: $\gamma = \varphi[x := d^{k+1}_i]$

Перестроим доказательство $M_k\vdash \gamma\to W$:
заменим во всём доказательстве $d^{k+1}_i$ на $y$.
Коллизий нет: под квантором $d^{k+1}_i$ не стоит, переменной не является. 
Получим доказательство $M_k\vdash \gamma[d^{k+1}_i := y]\to W$ и дополним его:

\begin{tabular}{ll}
$\varphi[x := y]\to W$ & $\varphi[x := d^{k+1}_i][d^{k+1}_i := y]$\\
$(\exists y.\varphi[x:=y])\to W$ & $y$ не входит в $W$ \\
$(\exists x.\varphi)\to(\exists y.\varphi[x:=y])$ & доказуемо (упражнение)\\
 \dots \\
$(\exists x.\varphi)\to W$ & доказуемо как $(\alpha\to\beta)\to(\beta\to\gamma)\vdash\alpha\to\gamma$ \\
$\exists x.\varphi$ & гипотеза\\
$W$
\end{tabular}
\end{itemize}
\hfill Q.E.D.

\deff{def:} $M^* = \bigcup_k M_k$

\thmm{Теорема} $M^*$ непротиворечиво.

\textbf{Доказательство:}

 От противного: доказательство противоречия конечной длины, гипотезы лежат в максимальном $M_k$, тогда $M_k$ противоречив.

\deff{def:} $M^\text{Б}$ --- множество всех бескванторных формул из $M^*$.

По непротиворечивому множеству $M$ можем построить $M^\text{Б}$ и для него построить модель $\mathcal{M}$.
Покажем, что эта модель годится для $M^*$ (и для $M$, так как $M \subset M^*$).

\deff{Лемма.}$\mathcal{M}$ есть модель для $M^*$.

\textbf{Доказательство}
Покажем, что при $\varphi\in M^*$ выполнено $\mathcal{M}\models\varphi$. Докажем индукцией по количеству кванторов в $\varphi$.
\begin{itemize}
\item База: $\varphi$ без кванторов. Тогда $\varphi\in M^\text{Б}$, отсюда $\mathcal{M}\models\varphi$ по построению $\mathcal{M}$.
\item Переход: пусть утверждение выполнено для всех формул с $n$ кванторами. Покажем, что это выполнено и для $n+1$ кванторов.
\begin{itemize}
\item Рассмотрим $\varphi = \exists x.\psi$, случай квантор всеобщности --- аналогично.

\item Раз $\exists x.\psi \in M^*$, то существует $k$, что $\exists x.\psi \in M_k$.
\item Значит, $\psi[x := d^{k+1}_i] \in M_{k+1}$. 
\item По индукционному предположению, $\mathcal{M}\models\psi[x := d^{k+1}_i]$ --- в формуле $n$ кванторов.
\item Но тогда $\llbracket \psi \rrbracket^{x := \llbracket d^{k+1}_i\rrbracket} = \text{И}$.
\item Отсюда $\mathcal{M}\models\exists x.\psi$.
\end{itemize}
\end{itemize}
\hfill Q.E.D.

\textbf{Формулировка и доказательство теоремы Гёделя}

\thmm{Теорема Гёделя о полноте исчисления предикатов}

Если $M$ --- замкнутое непротиворечивое множество формул, то оно имеет модель.

\textbf{Доказательство:}
\begin{itemize}
\item Построим по $M$ множество формул с поверхностными кванторами $M'$.
\item По $M'$ построим непротиворечивое множество замкнутых бескванторных формул $M^\text{Б}$ ($M^\text{Б}\subseteq M^*$, теорема о непротиворечивости $M^*$).
\item Дополним его до полного, построим для него модель $\mathcal{M}$ (теорема о существовании модели).
\item $\mathcal{M}$ будет моделью и для $M'$ ($M'\subseteq M^*$, лемма о модели для $M^*$), и, очевидно, для $M$.
\end{itemize}


\subsection{Полнота исчисления предикатов}

\thmm{Следствие из теоремы Гёделя о полноте}
Исчисление предикатов полно.

\textbf{Доказательство:}
\begin{itemize}
\item Пусть это не так, и существует формула $\varphi$, что $\models\varphi$, но $\not\vdash\varphi$.
\item Тогда рассмотрим $M = \{\neg\varphi\}$. 
\item $M$ непротиворечиво: если $\neg\varphi \vdash A\with\neg A$, то $\vdash \varphi$ (упражнение).
\item Значит, у $M$ есть модель $\mathcal{M}$, и $\mathcal{M}\models\neg\varphi$. 
\item Значит, $\llbracket \neg\varphi \rrbracket = \text{И}$, поэтому $\llbracket \varphi \rrbracket = \text{Л}$,
поэтому $\not\models\varphi$. Противоречие.
\end{itemize}


\subsection{Непротиворечивость исчисления предикатов}

\thmm{Теорема.} Если у множества формул $M$ есть модель $\mathcal{M}$, оно непротиворечиво. 

\textbf{Доказательство:}

Пусть противоречиво: $M\vdash A\with\neg A$, в доказательстве использованы гипотезы
$\delta_1, \delta_2,\dots,\delta_n$. Тогда $\vdash \delta_1\to\delta_2\to\dots\to\delta_n\to A\with\neg A$,
то есть $\llbracket\delta_1\to\delta_2\to\dots\to\delta_n\to A\with\neg A\rrbracket = \text{И}$ (корректность).
Поскольку все $\llbracket \delta_i \rrbracket_\mathcal{M} = \text{И}$, то
и $\llbracket A\with\neg A\rrbracket_\mathcal{M} = \text{И}$ (анализ таблицы истинности импликации). 
Однако $\llbracket A \with\neg A \rrbracket = \text{Л}$. Противоречие.

\textbf{Следствие:} Исчисление предикатов непротиворечиво 


\textbf{Доказательство:}

Рассмотрим $M = \varnothing$ и любую классическую модель.

Доказательства опираются на непротиворечивость метатеории.

\hfill Q.E.D.

\subsection{Теорема Гёделя о компактности}

Если $\Gamma$ --- некоторое семейство бескванторных формул, то $\Gamma$ имеет модель
тогда и только тогда, когда любое его конечное подмножество имеет модель.

\textbf{Доказательство:}

$(\Rightarrow)$: очевидно

$(\Leftarrow)$: пусть каждое конечное подмножество имеет модель. Тогда $\Gamma$ непротиворечиво:

{\footnotesize Иначе для любой $\sigma$ выполнено $\Gamma\vdash\sigma$. В частности, для $\gamma\in\Gamma$
выполнено $\Gamma\vdash\neg\gamma$. Доказательство имеет конечную длину и использует конечное
количество формул $\gamma_1,\dots,\gamma_n\in\Gamma$. Тогда рассмотрим $\Sigma = \{\gamma,\gamma_1,\dots,\gamma_n\}$
и модель $\mathcal{S}$ для неё. Тогда:
\begin{enumerate}
\item $\models_{S}\gamma$ (определение модели)
\item $\models_{S}\neg\gamma$ (теорема о корректности: $\Sigma\vdash\neg\gamma$, значит $\Sigma\models\neg\gamma$ в любой модели)
\end{enumerate}}

Значит, $\Gamma$ имеет модель (вспомогательная теорема к теореме Гёделя о полноте).

\hfill Q.E.D.   