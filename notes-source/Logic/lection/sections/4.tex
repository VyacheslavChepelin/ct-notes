
\section{Лекция 4.}

\subsection{Модели Крипке}

\deff{def:} Модель Крипке $\langle \mathcal{W}, (\preceq), (\Vdash)\rangle$:
\begin{itemize}
\item $\mathcal{W}$ --- множество миров, $(\preceq)$ --- нестрогий частичный порядок на $\mathcal{W}$;
\item $(\Vdash)\subseteq \mathcal{W}\times P$ --- отношение вынуждения
между мирами и переменными, причём, если $W_i \preceq W_j$ и $W_i \Vdash X$, то $W_j \Vdash X$.
\end{itemize}

Доопределим вынужденность:
\begin{itemize}
\item $W \Vdash \alpha\with\beta$, если $W \Vdash \alpha$ и $W \Vdash \beta$;
\item $W \Vdash \alpha\vee\beta$, если $W \Vdash \alpha$ или $W \Vdash \beta$;
\item $W \Vdash \alpha\rightarrow\beta$, если всегда при $W \preceq W_1$ и $W_1 \Vdash \alpha$ выполнено $W_1 \Vdash \beta$
\item $W \Vdash \neg\alpha$, если всегда при $W \preceq W_1$ выполнено $W_1 \not\Vdash \alpha$.
\end{itemize}

Будем говорить, что $\Vdash\alpha$, если $W\Vdash\alpha$ при всех $W \in \mathcal{W}$.
Будем говорить, что $\models_\kappa\alpha$, если $\Vdash\alpha$ во всех моделях Крипке.

\deff{Корректность моделей Крипке}

\thmm{Лемма.} Если $W_1 \Vdash \alpha$ и $W_1 \preceq W_2$, то $W_2 \Vdash \alpha$

\thmm{Теорема.}

Пусть $\langle \mathcal{W}, (\preceq), (\Vdash)\rangle$ ---
некоторая модель Крипке.
Тогда она есть корректная модель интуиционистского исчисления высказываний.

\textbf{Доказательство:}

Доказательство для древовидного $(\preceq)$, обобщение на произвольный порядок легко построить.

Заметим, что $V(\alpha) := \{ w \in \mathcal{W}\ |\ w\Vdash\alpha\}$ открыто в топологии для деревьев.
Значит, положив $V = \{\ S\ |\ S \subseteq \mathcal{W}\ \with\ S \text{ --- открыто }\}$ и
$\llbracket \alpha \rrbracket = V(\alpha)$, получим алгебру Гейтинга.

\hfill Q.E.D.

\subsection{Табличные модели}

\deff{def:} 
Пусть задано $V$, значение $T \in V$ (<<истина>>), функция $f_P: P \rightarrow V$, 
функции $f_\with, f_\vee, f_\rightarrow : V \times V \rightarrow V$,
функция $f_\neg: V \rightarrow V$.

Тогда оценка $\llbracket X \rrbracket = f_P(X)$, 
$\llbracket \alpha\star\beta \rrbracket = f_\star(\llbracket \alpha \rrbracket, \llbracket \beta \rrbracket)$,
$\llbracket \neg\alpha \rrbracket = f_\neg(\llbracket\alpha\rrbracket)$ --- \textbf{табличная}.

Если $\vdash \alpha$ влечёт $\llbracket\alpha\rrbracket = T$ при всех оценках пропозициональных переменных $f_P$, 
то $\mathcal{M} := \langle V, T, f_\with, f_\vee, f_\rightarrow, f_\neg\rangle$ --- \textbf{табличная модель}.


\deff{def:} Табличная модель \textbf{конечна}, если $V$ конечно.

\thmm{Теорема.}

Не существует полной конечной табличной модели для интуиционистского исчисления высказываний

\deff{Доказательство нетабличности: $\alpha_n$}

Пусть существует полная конечная табличная модель $\mathcal{M}$, $V = \{v_1, v_2, \dots, v_n\}$.
То есть, если $\models_\mathcal{M}\alpha$, то $\vdash\alpha$.

Рассмотрим $$\alpha_n = 
            \bigvee_{1 \le p < q \le n+1} A_p \rightarrow A_q
           $$
Рассмотрим оценку $f_P: \{A_1 \dots A_{n+1}\} \rightarrow \{v_1 \dots v_n\}$.
По принципу Дирихле существуют $p \ne q$, что $\llbracket A_p \rrbracket = \llbracket A_q \rrbracket$.
Значит, $$\llbracket A_p \rightarrow A_q \rrbracket = f_\rightarrow (\llbracket A_p \rrbracket, \llbracket A_q \rrbracket) = f_\rightarrow(v,v)$$
С другой стороны, $\vdash X \rightarrow X$ --- поэтому $f_\rightarrow(\llbracket X \rrbracket, \llbracket X \rrbracket) = T$,
значит, $$\llbracket A_p \rightarrow A_q \rrbracket = f_\rightarrow(v,v) = f_\rightarrow(\llbracket X \rrbracket, \llbracket X \rrbracket) = T$$

Аналогично, $\vdash \sigma \vee (X \rightarrow X) \vee \tau$, отсюда $\llbracket \alpha_n \rrbracket = \llbracket \sigma \vee (X \rightarrow X) \vee \tau \rrbracket = T$.

Однако, в такой модели $\not\Vdash \alpha_n$:

\begin{center}\tikz{
\node at (0,0)   (R) {$W_R$};
\node at (3,1.5) (A1) {$W_1$}; \node[right] at (3.5,1.5) (A11) {\color{black!50!red} $\Vdash A_1$};
  \draw[red,fill=red,opacity=0.2](A1.south west) 
     to[closed,curve through={($(A1.south west)!0.5!(A1.south east)$) .. (A1.north east)}] (A1.north west);

\node at (3,0.5) (A2) {$W_2$}; \node[right] at (3.5,0.5) (A21) {\color{black!50!magenta} $\Vdash A_2$};
\draw[red,fill=magenta,opacity=0.2](A2.north west) 
     to[closed,curve through={(A2.south west) .. (A2.south east)}] (A2.north east);
\node at (3,-0.2) (A3) {$\dots$};
\node at (3,-0.9) (A4) {$W_n$}; \node[right] at (3.5,-0.9) (A41) {\color{teal} $\Vdash A_n$};
\draw[red,fill=teal,opacity=0.2](A4.north west) 
     to[closed,curve through={($(A4.north west)!0.5!(A4.north east)$) .. (A4.south east)}] (A4.south west);

\draw[->] (R) to (A1); 
\draw[->] (R) to (A2); 
\draw[->] (R) to (A4); 

\node[right] at (6,1.5) {Если $q > 1$, то}; \node[right] at (8.6, 1.5) {$W_1 \not\Vdash A_q$ и $W_1 \not\Vdash A_1 \rightarrow A_q$};
\node[right] at (6,0.5) {Если $q > 2$, то}; \node[right] at (8.6, 0.5) { $W_2 \not\Vdash A_q$ и $W_2 \not\Vdash A_2 \rightarrow A_q$};
\node[right] at (8.6,-0.5) {$W_n \not\Vdash A_{n+1}$; $W_n \not\Vdash A_n \rightarrow A_{n+1}$};
\node[right] at (6,-1.5) {Если $p < q$, то}; \node[right] at (8.6, -1.5) { $W_p \not\Vdash A_q$ и $W_p \not\Vdash A_p \rightarrow A_q$};
}
\end{center}

Если $p < q$, то $W_p \not\Vdash A_p \rightarrow A_q$, то есть $W_R \not\Vdash A_p \rightarrow A_q$.

Отсюда: $W_R \not\Vdash \bigvee_{p < q} A_p \rightarrow A_q$, $W_R \not\Vdash \alpha_n$,
 потому $\not\models \alpha_n$ и $\not\vdash \alpha_n$.

\hfill Q.E.D.

\deff{def:} Исчисление \textbf{дизъюнктивно}, если при любых $\alpha$ и $\beta$ из $\vdash\alpha\vee\beta$ следует $\vdash\alpha$ или $\vdash\beta$.

\deff{def:} \textbf{Решётка гёделева}, если $a + b = 1$ влечёт $a = 1$ или $b = 1$.

\thmm{Теорема}

Интуиционистское исчисление высказываний дизъюнктивно

\subsection{Алгебра Линденбаума как псевдобулева алгебра}

\begin{itemize}
\item \emph{(импликативная ...)} Покажем $[\alpha]\rightarrow[\beta] = [\alpha\rightarrow\beta]$:

в самом деле, $[\alpha]\rightarrow[\beta] = \text{наиб }\{[\sigma]\ |\ [\alpha\with\sigma] \le [\beta]\}$. Покажем требуемое
двумя включениями:

\begin{enumerate}
\item $\alpha\with(\alpha\rightarrow\beta) \vdash \beta$ (карринг + транзитивность импликации)
\item Если $\alpha\with\sigma \vdash \beta$, то $\sigma\vdash\alpha\rightarrow\beta$ (карринг + теорема о дедукции)
\end{enumerate}

\item \emph{(... с нулём ...)} Покажем, что $0 = [A \with \neg A]$: 

в самом деле, $A \with \neg A \vdash \sigma$ при любом $\sigma$.

\item \emph{(... согласованная с ИИВ)}
\begin{enumerate}
\item Из доказательства видно, что $[\alpha\with\beta] = [\alpha]\cdot[\beta]$,
$[\alpha\vee\beta] = [\alpha]+[\beta]$, $[\alpha\rightarrow\beta]=[\alpha]\rightarrow[\beta]$, $[A\with\neg A] = 0$.

\item $[A \rightarrow A] = [A] \rightarrow [A] = 1$ по свойствам алгебры Гейтинга
\item $[\neg \alpha] = [\alpha \rightarrow A\with\neg A] = [\alpha] \rightarrow 0 = \sim[\alpha]$
\end{enumerate}
\end{itemize}

\subsection{Гёделевизация (операция $\Gamma(\mathcal{A})$)}

\deff{def:} Для алгебры Гейтинга $\mathcal{A} = \langle A, (\preceq) \rangle$ определим операцию \textbf{<<гёделевизации>>}: 
$\Gamma(\mathcal{A}) = \langle A\cup\{\omega\}, (\preceq_{\Gamma(\mathcal{A})}) \rangle$, где
отношение $(\preceq_{\Gamma(\mathcal{A})})$ --- минимальное отношение порядка,
удовлетворяющее условиям:

\vspace{-0.5cm}
\begin{center}\begin{tabular}{cc}
\begin{minipage}{9cm}
\begin{itemize}
\item $a \preceq_{\Gamma(\mathcal{A})} b$, если $a \preceq_\mathcal{A} b$ и $a,b \notin \{\omega,1\}$;
\item $a \preceq_{\Gamma(\mathcal{A})} \omega$, если $a \ne 1$;
\item $\omega \preceq_{\Gamma(\mathcal{A})} 1$
\end{itemize}
\end{minipage}
&
\begin{minipage}{4cm}\begin{center}
\tikz{
    \filldraw[pattern=north west lines,pattern color=gray] (1,-1) circle (1cm);
    \node[right] at (2.2,-1) (A) {$A \setminus \{1\}$};
    \node[circle,fill,inner sep=2pt, outer sep=0pt,label=right:$1$] at (1,1) (Max) {};
    \node[circle,fill,inner sep=2pt, outer sep=0pt,label=above right:$\omega$] at (1,0) (Omega) {}; 
    \draw[-stealth,line width=1] (Max) to (Omega);
}\end{center}
\end{minipage}
\end{tabular}\end{center}


\thmm{Теорема.}

$\Gamma(\mathcal{A})$ --- гёделева алгебра.

\textbf{Доказательство:}

Проверка определения алгебры Гейтинга и наблюдение: если $a \preceq \omega$ и $b \preceq \omega$, то $a + b \preceq \omega$.

\hfill Q.E.D.

\deff{def:}
Определим $\llbracket\cdot\rrbracket_{\Gamma(\mathcal{L})} : \mathcal{F} \rightarrow \Gamma(\mathcal{L})$.
Положим $\llbracket X \rrbracket_{\Gamma(\mathcal{L})} := \llbracket X \rrbracket_\mathcal{L}$.
Связки определим естественным образом:
$\llbracket \alpha\with\beta \rrbracket_{\Gamma(\mathcal{L})} := \llbracket \alpha\rrbracket_{\Gamma(\mathcal{L})}\cdot\llbracket\beta \rrbracket_{\Gamma(\mathcal{L})}$
и т.п.

\thmm{Теорема.}

Оценка является алгеброй Гейтинга, согласованной с ИИВ.


\textbf{Доказательство:}

$\Gamma(\mathcal{L})$ --- алгебра Гейтинга.
Также заметим, что:
\begin{itemize}
\item $\llbracket \alpha\rightarrow\alpha \rrbracket_{\Gamma(\mathcal{L})} = \llbracket \alpha\rrbracket_{\Gamma(\mathcal{L})}\rightarrow\llbracket\alpha \rrbracket_{\Gamma(\mathcal{L})}
= 1_{\Gamma(\mathcal{L})}$
\item $\llbracket \alpha\with\neg\alpha \rrbracket_{\Gamma(\mathcal{L})} = \llbracket \alpha\rrbracket_{\Gamma(\mathcal{L})} \cdot (\llbracket \alpha\rrbracket_{\Gamma(\mathcal{L})}\rightarrow 0)
= 0_{\Gamma(\mathcal{L})}$.
\end{itemize}

Согласованность оценки следует из определения и указанных выше соображений.

\hfill Q.E.D.

\subsection{Гомоморфизм алгебр}

\deff{def:} Пусть $\mathcal{A}, \mathcal{B}$ --- \textbf{алгебры Гейтинга}. Тогда $g: \mathcal{A} \rightarrow \mathcal{B}$ --- гомоморфизм,
если $g(a \star b) = g(a) \star g(b)$, $g(0_\mathcal{A}) = 0_\mathcal{B}$ и $g(1_\mathcal{A}) = 1_\mathcal{B}$.

\deff{def:} Будем говорить, что оценка $\llbracket\cdot\rrbracket_\mathcal{A}$ согласована
с $\llbracket\cdot\rrbracket_\mathcal{B}$ и гомоморфизмом $g$, если $g(\mathcal{A}) = \mathcal{B}$ и
$g(\llbracket\alpha\rrbracket_\mathcal{A}) = \llbracket\alpha\rrbracket_\mathcal{B}$.


\deff{def: }[$\mathcal{G}: \Gamma(\mathcal{L}) \rightarrow \mathcal{L}$]\vspace{-0.5cm}
$$\mathcal{G}(a) = \left\{\begin{array}{ll} a, & a \ne \omega\\
                                  1, & a = \omega\end{array}\right.$$\vspace{-0.5cm}

\thmm{Лемма} 

$\mathcal{G}$ --- гомоморфизм $\Gamma(\mathcal{L})$ и $\mathcal{L}$, причём 
оценка $\llbracket\cdot\rrbracket_{\Gamma(\mathcal{L})}$ согласована с $\mathcal{G}$
и $\llbracket\cdot\rrbracket_\mathcal{L}$.

\thmm{Теорема}

Если $\vdash \alpha\vee\beta$, то либо $\vdash\alpha$, либо $\vdash\beta$.

\textbf{Доказательство:}

Пусть $\vdash\alpha\vee\beta$. Тогда $\llbracket\alpha\vee\beta\rrbracket_{\Gamma(\mathcal{L})} = 1$
(так как данная оценка согласована с ИИВ). Тогда $\llbracket\alpha\rrbracket_{\Gamma(\mathcal{L})} = 1$ или
$\llbracket\beta\rrbracket_{\Gamma(\mathcal{L})} = 1$ (так как $\Gamma(\mathcal{L})$ гёделева). 

Пусть $\llbracket\alpha\rrbracket_{\Gamma(\mathcal{L})} = 1$, 
тогда $\mathcal{G}(\llbracket\alpha\rrbracket_{\Gamma(\mathcal{L})}) = \llbracket\alpha\rrbracket_\mathcal{L} = 1$, 
тогда $\vdash\alpha$ (по полноте $\mathcal{L}$).

\hfill Q.E.D.

\subsection{Построение дистрибутивных подрешёток}

\deff{def:} Решётка $\mathcal{L'} = \langle L', (\preceq) \rangle$ --- подрешётка решётки $\mathcal{L} = \langle L, (\preceq) \rangle$, 
если $L' \subseteq L$, $(\preceq') \subseteq (\preceq)$ и 
при $a,b \in L'$ выполнено $a +_{\mathcal{L'}} b = a +_{\mathcal{L}} b$ и $a \cdot_{\mathcal{L'}} b = a \cdot_{\mathcal{L}} b$.

\thmm{Лемма.} 

Существует дистрибутивная подрешётка $\mathcal{L'}$, содержащая
$a_1, \dots, a_n$, что $|L'| \le 2^{2^n}$.


\textbf{Доказательство:}

Пусть $\mathcal{L'} = \langle\{ \varphi(a_1,\dots,a_n)\ |\ \varphi \text{ составлено из (+) и }(\cdot)\}, (\preceq)\rangle$.
Заметим, что если $p,q \in L'$, то $p \star_{\mathcal{L}} q \in L'$ 
(так как $\varphi_p(\overrightarrow{a})\star\varphi_q(\overrightarrow{a}) = \psi(\overrightarrow{a})$). Также ясно,
что если $\sup_L\{p,q\} \in L'$ (или $\inf_L\{p,q\} \in L'$), то $p \star_{\mathcal{L}} q = p \star_{\mathcal{L'}} q$.
Значит, $\mathcal{L'}$ также дистрибутивна. Построим <<ДНФ>>:
$$\varphi(a_1,\dots,a_n) = \sum_{\text{Кн} \in \text{ДНФ}(\varphi)}\prod_{i \in \text{Кн}}a_i$$

Всего не больше $2^n$ возможных компонент и $2^{2^n}$ возможных формул $\varphi(\overrightarrow{a})$.

\hfill Q.E.D


\thmm{Теорема}

Если $\not\vdash \alpha$ в ИИВ, то существует $\mathcal{G}$,
что $\mathcal{G} \not\models \alpha$, причём $|\mathcal{G}| \le 2^{2^{|\alpha|+2}}$.

\textbf{Доказательство:}

Если $\not\vdash \alpha$, то по 
полноте найдётся алгебра Гейтинга $\mathcal{H}$, что
$\mathcal{H} \not\models \alpha$. 

Пусть $\varphi_1, \dots, \varphi_n$ --- подформулы $\alpha$.
Пусть $\mathcal{G}$ --- дистрибутивная подрешётка $\mathcal{H}$, 
построенная по $\llbracket \varphi_1 \rrbracket, \dots, \llbracket \varphi_n \rrbracket$, $0$ и $1$. 

Очевидно, что $\mathcal{G}$ --- алгебра Гейтинга, и можно показать, 
что $\mathcal{G} \not\models \alpha$ (псевдодополнения не обязаны сохраниться).
Тогда по лемме, $|\mathcal{G}| \le 2^{2^{n+2}}$. 


\thmm{Теория}

ИИВ разрешимо.

\textbf{Доказательство:}

По формуле $\alpha$ построим все возможные алгебры Гейтинга $\mathcal{G}$ размера не больше $2^{2^{|\alpha|+2}}$,
если $\mathcal{G}\models\alpha$, то $\vdash\alpha$.

\hfill Q.E.D.


