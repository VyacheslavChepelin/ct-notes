\subsection{Задача 1.}

Покажите, что в классическом исчислении высказываний $\Gamma \models \alpha \Rightarrow \Gamma \vdash \alpha$.

\deff{Решение:}



Пусть $X_1,\ldots, X_n$ - пропозициональные переменные, которые участвуют в формуле $\alpha$, их конечное количество.

Посмотрю на таблицу истинности ($2^n$ строк оценки), для $\alpha$. 

Для каждой выполненной строчки выполнено:

$T_1 \land \ldots \land T_n \models \alpha$, где $T_i = X_i $ или $T_i = \neg X_i$, в зависимости


 Это эквивалентно тому, что формула  
 верна $\models (T_1 \land \dots \land T_n) \to \alpha$.  Откуда по теореме о полноте из лекции:
\[
\vdash(T_1 \land T_2 \land \dots \land T_n) \to \alpha
\]
Используем теорему о дедукции и получим:
\[
(T_1 \land T_2 \land \dots \land T_n) \vdash \alpha
\]



Заметим, что $\Gamma \models (T_1 \land T_2 \land \dots \land T_n) \lor (Q_1 \ldots) \lor \ldots$ ---выводит какое-то конечное количество множество строк нашей таблицы истинности

Осталось показать, что
\[
\Gamma \vdash (T_1 \land T_2 \land \dots \land T_n) \lor (Q_1 \ldots) \lor \ldots
\]
И тогда будет выполнено $\Gamma \vdash \alpha$

Обозначу $(T_1 \land T_2 \land \dots \land T_n) \lor (Q_1 \ldots) \lor \ldots =D$

Формула $D$ построена так, что она "перечисляет" все возможные строки $\Gamma$. В классической логике это эквивалентно тому, что $\Gamma$  эквивалентно $D$ (в плане таблицы истинности)


\newpage

\subsection{Задача 2.}

а) Возьмем открытое множество из $R$, по определению вокруг каждой точки открытого множества есть шар, возьмем объедение и победим (будем рассматривать только $\mathbb{Q}$, а так как $Q$ плотно в $\mathbb{R}$ и так как $\mathbb{Q}$ - счетно, то покрытие будет счетно и по определению мы сможем взять объединение) TODO: расписать


б) нет

в) нет