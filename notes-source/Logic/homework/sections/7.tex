\subsection{ № 5}

Рассмотрим аксиоматику Пеано. 
Пусть $$a^b = \left\{\begin{array}{ll}1,& b= 0 \\a^c\cdot a,&b = c'\end{array}\right.$$

\textbf{Сложение}:
$$
\begin{cases}
   a + 0 = a\\
    a + b' = (a + b)'
\end{cases}
$$
\textbf{Умножение}:
$$
\begin{cases}
    a \cdot 0 = 0\\
     a \cdot b' = a \cdot b + a
\end{cases}
$$

Докажите, что:
\begin{enumerate}
\item $a \cdot b = b \cdot a$



\item $(a + b) \cdot c = a \cdot c + b \cdot c$
\item $a^{b+c} = a^b \cdot a^c$
\item $(a^b)^c = a^{b \cdot c}$
\item $(a + b) + c = a + (b + c)$
\end{enumerate}


\textbf{Доказательство}

5.\((a + b) + c = a + (b + c)\)

\textbf{Доказательство индукцией по \( c \):}

\textbf{База:} \( c = 0 \)
  \[
  (a + b) + 0 = a + b = a + (b + 0)
  \]

\textbf{Предположение индукции:} 
  \[
  (a + b) + c = a + (b + c)
  \]

\textbf{Шаг индукции:} \( c \to c' \)
  \[
  \begin{aligned}
  (a + b) + c' &= ((a + b) + c)' = (a + (b + c))' \\
  a + (b + c') &= a + (b + c)' = (a + (b + c))'
  \end{aligned}
  \]


1. \( a \cdot b = b \cdot a \)

\textbf{Докажем вспомогательное утверждение}

$$b'\cdot a=b \cdot a + a $$

\textbf{База:} \( a = 0 \)
  \[
  b' \cdot 0 = 0
  \]
  очевидно

\textbf{Предположение:} \( b'\cdot a=b \cdot a + a \)

\textbf{Шаг:} \( a \to a' \)
  \[
  b'\cdot a'=b' \cdot a + b'=b \cdot a + a + b' =  \]
  $$= b \cdot a + (a+b)' =  b \cdot a + (b + a)' = b \cdot a + b + a' = b \cdot a' + a'$$

Доказано вернемся к искомой задаче

\textbf{База:} \( b = 0 \)
  \[
  a \cdot 0 = 0 = 0 \cdot a
  \]
  очевидно

\textbf{Предположение:} \( a \cdot b = b \cdot a \)

\textbf{Шаг:} \( b \to b' \)
  \[
  \begin{aligned}
  a \cdot b' = a \cdot b + a = b \cdot a + a = b'\cdot a
  \end{aligned}
  \]




 2.\((a + b) \cdot c = a \cdot c + b \cdot c\)

\textbf{База:} \( c = 0 \)
  \[
  (a + b) \cdot 0 = 0 = 0 + 0 = a \cdot 0 + b \cdot 0
  \]

\textbf{Предположение:}
  \[
  (a + b) \cdot c = a \cdot c + b \cdot c
  \]

\textbf{Шаг: \( c \to c' \)}
  \[
  (a + b) \cdot c' = (a + b) \cdot c + (a + b) \\
  = (a \cdot c + b \cdot c) + (a + b) \]
  \[
  a \cdot c' + b \cdot c' = (a \cdot c + a) + (b \cdot c + b) \\
  = a \cdot c + b \cdot c + a + b
  \]




3. \( a^{b + c} = a^b \cdot a^c \)



\textbf{База:} \( c = 0 \)
  \[
  a^{b + 0} = a^b = a^b \cdot 1 = a^b \cdot a^0
  \]
\textbf{Предположение:} 
  \[
  a^{b + c} = a^b \cdot a^c
  \]

\textbf{Шаг:} \( c \to c' \)
  $$
  a^{b + c'} = a^{(b + c)'} = a^{b + c} \cdot a = (a^b \cdot a^c) \cdot a$$ 
  $$ 
  a^b \cdot a^{c'} = a^b \cdot (a^c \cdot a) = (a^b \cdot a^c) \cdot a$$

  

4. \( (a^b)^c = a^{b \cdot c} \)



\textbf{База:} \( c = 0 \)
  \[
  (a^b)^0 = 1 = a^0 = a^{b \cdot 0}
  \]
\textbf{Предположение:} 
  \[
  (a^b)^c = a^{b \cdot c}
  \]
\textbf{Шаг:} \( c \to c' \)
  \[
  (a^b)^{c'} = (a^b)^c \cdot a^b = a^{b \cdot c} \cdot a^b \]
  \[
  a^{b \cdot c'} = a^{b \cdot c + b} = a^{b \cdot c} \cdot a^b
  \]
  
\newpage
\subsection{№ 6}

Определим отношение <<меньше или равно>> так: $0 \le a$ и $a' \le b'$, если $a \le b$. Докажите, что:
\begin{enumerate}
\item $x \le x+y$;
\item $x \le x \cdot y$ (укажите, когда это так --- в остальных случаях приведите контрпримеры);
\item Если $a \le b$ и $m \le n$, то $a \cdot m \le b \cdot n$;
\item $x \le y$ тогда и только тогда, когда существует $n$, что $x + n = y$;
\item Будем говорить, что $a$ делится на $b$ с остатком, если существуют такие $p$ и $q$, что 
$a = b \cdot p + q$ и $0 \le q < b$. Покажите, что $p$ и $q$ всегда существуют и единственны,
если $b > 0$.
\end{enumerate}

$a \leq a'$ - очевидно


1. \( x \le x + y \)



\textbf{База:} \( y = 0 \)  
  \( x + 0 = x \), очевидно \( x \le x \). 

\textbf{Предположение индукции:} \( x \le x + y \)

\textbf{Шаг индукции:} \( y \to y' \)  
  \[
  x + y' = (x + y)'
  \]  
  Из предположения \( x \le x + y \) и  \( a \le a' \):  
  \[
  x + y \le (x + y)' = x + y'
  \]  


2. \( x \le x \cdot y \)

\( x \le x \cdot y \) выполняется тогда и только тогда, когда \( x = 0 \) или \( y \ge 1 \).

\textbf{Доказательство:}

Если \( x = 0 \):  
  \( 0 \cdot y = 0 \), и \( 0 \le 0 \). 

 Если \( y = 1 \):  
  \( x \cdot 1 = x \), и \( x \le x \). 

 Если \( y \ge 1 \):  

  \textbf{База:} \( y = 1 \) — уже проверено.  
  
  \textbf{Предположение:} \( x \le x \cdot y \)
  
  \textbf{Шаг:} \( y \to y' \)  
  \[
  x \cdot y' = x \cdot y + x
  \]  
  \[
  x \le x \cdot y 
  \]
  По пункту 1.
  \[
  x \le x \cdot y + x = x \cdot y'
  \]


\textbf{Контрпример:}

При \( x = a \), \( y = 0 \):  
\( a \cdot 0 = 0 \), но \( a \nleq 0 \).



 3. Если \( a \le b \) и \( m \le n \), то \( a \cdot m \le b \cdot n \)

\textbf{Доказательство:}

Из пункта 4:  
\[
\begin{aligned}
a \le b &\Leftrightarrow \exists k : b = a + k \\
m \le n &\Leftrightarrow \exists l : n = m + l
\end{aligned}
\]  
Тогда:  
\[
b \cdot n = (a + k)(m + l) \]
\[ a \cdot m + a \cdot l + k \cdot m + k \cdot l
\]  
Следовательно,  
\[
b \cdot n = a \cdot m + (a \cdot l + k \cdot m + k \cdot l)
\]  
По пункту 4: \( a \cdot m \le b \cdot n \).

4. \( x \le y \iff \exists n : x + n = y \)

\textbf{Доказательство:}

\( \Rightarrow \):  
  \begin{enumerate}
      \item База: Если \( x = 0 \), то \( n = y \). 
      \item Если \( x = a' \), \( y = b' \) и \( a \le b \), то по предположению \( \exists n : a + n = b \). 
    
    Тогда:  
    \[
    b' = (a + n)' = a' + n \Rightarrow x + n = y
    \]  
  \end{enumerate}


\( \Leftarrow \)
\begin{enumerate}
    \item База: $$n = 0 :  x + 0 = x \Rightarrow x \le x $$
    \item По предположению: \( x \le x + n \).  
    Так как \(x \leq  x + n, x\le x + n' = y \) по пункту 1
 
\end{enumerate}


\newpage
\subsection{№ 7}

Обозначим за $\overline{n}$ представление числа $n$ в формальной арифметике: %, по сути это ноль с $n$ штрихами:

$$\overline{n} = \left\{\begin{array}{ll}0, &n = 0\\
           \left(\overline{k}\right)', & n=k+1\end{array}\right.$$

Например, $\overline{5} = 0'''''$. Докажите в формальной арифметике (доказательства могут использовать
метаязык, но при этом из текста должно быть понятно, как выстроить полное доказательство):
\begin{enumerate}
\item $\vdash \overline{2} \cdot \overline{2} = \overline{4}$;

\item $\vdash \forall a.a \cdot 0 = 0 \cdot a$;
\item $\vdash \forall a.a \cdot \overline{2} = a + a$;
\item $\vdash \forall p.(\exists q.q' = p) \vee p = 0$ (единственность нуля);
\end{enumerate}


1. $\vdash \overline{2} \cdot \overline{2} = \overline{4}$

Используем аксиомы умножения и сложения:
\begin{enumerate}
    \item $\overline{2} = 0''$, $\overline{1} = 0'$, $\overline{0} = 0$.
    \item По аксиоме умножения: $a \cdot b' = a \cdot b + a$.
    \item По аксиоме сложения: $a + b' = (a + b)'$.
\end{enumerate}

Вычисляем:

1. $\overline{2} \cdot \overline{1} = \overline{2} \cdot \overline{0}' = \overline{2} \cdot \overline{0} + \overline{2} = 0 + \overline{2} = \overline{2}$ (так как $0 + a = a$ — ранее доказанное свойство).

2. $\overline{2} \cdot \overline{2} = \overline{2} \cdot \overline{1}' = \overline{2} \cdot \overline{1} + \overline{2} = \overline{2} + \overline{2}$.

3. $\overline{2} + \overline{2} = \overline{2} + \overline{1}' = (\overline{2} + \overline{1})' = \overline{3}' = \overline{4}$.

Таким образом, $\overline{2} \cdot \overline{2} = \overline{4}$. 

2. $\vdash \forall a.a \cdot 0 = 0 \cdot a$;

\textbf{Лемма 2:} $ a = b \vdash b = a$

\begin{tabular}{lll}
     (1)&$a =b $& Гипотеза \\
     (2)&$a =a $& из лекции \\
     (3)&$a =b \rightarrow a = a \rightarrow b = a $& А1 \\
     (4)&$b= a$ & MP (1,2),3
\end{tabular}

\newpage
\textbf{Лемма 2:} $0 \cdot a = 0$

\begin{tabular}{lll}
     (1)&$0 \cdot 0 = 0$& А7 \\
     (2)&$0 \cdot a' = 0 \cdot a + 0$ & А8\\
     (3)&$0 \cdot a + 0= 0 \cdot a $ & А7\\
     (4)&$0 \cdot a' = 0 \cdot a$ &  Из аксиомы 1, см доказательство(транзитивность =)\\
     (5)&$0 \cdot a = 0 \cdot a'$ &  по Лемме 1\\
     (6)& $0 \cdot a = 0 \cdot a' \rightarrow 0 \cdot a = 0 \rightarrow 0 =0 \cdot a'$ & A1\\
     (7)& $0 \cdot a = 0 \rightarrow 0 =0 \cdot a' $ & MP 5,6\\
     (7.5)& $\forall a.0 \cdot a = 0 \rightarrow 0 =0 \cdot a' $ & MP 5,6, обозначим это за T\\
     (8)& $\varphi[a:=0] \& (\forall a.\varphi \rightarrow \varphi[a := a']) \rightarrow \varphi$ & по схеме индукции, где $\varphi(a) := 0 \cdot a = 0$\\
     (9)&$0\cdot 0 = 0 \rightarrow T \rightarrow 0 \& T$& А4 из исчисления предикатов\\
     (10)& $T \rightarrow 0 \& T$ & MP 9,1\\
     (11)& $ 0 \& T$ & MP 7.5,10\\
     (12)& $\varphi$ & MP 11,8\\
\end{tabular}

\textbf{Доказательство}

\begin{tabular}{lll}
     (1) &$0 = a\cdot 0 \rightarrow 0 =0 \cdot  a \rightarrow a\cdot 0 = 0 \cdot a$& А1 \\
     (2) &$a\cdot 0 = 0 \rightarrow a\cdot 0 = a 
     \cdot 0 \rightarrow  0 = a \cdot 0$& А1   \\
     (3) &$a\cdot 0 = 0$& А7  \\
     (4) &$a\cdot 0 = a \cdot 0$& из лекции  \\
     (5) &$a\cdot a \cdot 0 \rightarrow  0 = a \cdot 0$ & MP 3,2 \\
     (6) &$0 = a \cdot 0$ & MP 4,5 \\
     (7) &$0 \cdot a = 0 $& по Лемме 2 \\
     (8) &$ 0= 0 \cdot a $& из Лемме 1 \\
     (9) &$0 =0 \cdot  a \rightarrow a\cdot 0 = 0 \cdot a$& MP 7,1\\
     (10)& $ a\cdot 0 = 0 \cdot a$& MP 8,9
\end{tabular}

\pagebreak

\subsection{№ 2}



