\section{Лекция 4.}

\subsection{Замена переменных дифференциальном уравнения.}

$x = p(u,v), y = q(u,v)$

Цель такой замены --- упростить и свести к известному виду.

Дифференциалы прежних переменных преобразуются по формулам:
$$dx = p'_u du + p'_v dv \quad dy = q'_u du + q'_v dv$$

\thmm{Теорема (замена переменных в ДУ)}

Пусть $G$ - область в $\mathbb{R}^2_{x,y}$. $\Phi: G \subset \mathbb{R}^2_{x,y}\xrightarrow{} \mathbb{R}_{u,v}^2$ --- диффеоморфизм, $F: G \rightarrow \mathbb{R}^2$. 
$$H = (F \circ \Phi^{-1})(\Phi^{-1})'$$
Тогда отображение $\Phi$ устанавливает взаимно-однозначное соответствие между интегральными кривыми уравнений:
$$F(r) d r = 0, r \in G$$
$$H(s) ds = 0, s \in \Phi(G)$$

\textbf{Замечание:} Это имеет такой смысл: у вас есть диффеоморфизм между двумя областями --- ваша функция замены переменных из $\Phi: x,y \rightarrow u,v $ мы берем обратную и производную и выигрываем

\subsection{Однородное уравнение}

\deff{def:} Функция $F(x.y)$ называется \textbf{однородной функцией} степени $\alpha$, если при всех допустимых $t,x,y$ верно равенство:
$$F(tx,ty) = t^{\alpha}F(x,y)$$
\deff{def:} Пусть $P,Q$ --- однородные функции одинаковой степени. Тогда уравнение вида:
$$P(x,y)dx + Q(x,y)dy = 0$$
называется \textbf{однородным уравнением}.

\textbf{Давайте сведем однородно уравнение к уравнением с разделяющимися переменными.}

\begin{enumerate}
    \item Сделаем замену $x =u, y = uv$

    \textbf{Замечание:} поскольку переменные $u$ и $x$  совпадают, то переменную $u$ обычно не вводят, а полагают:
    $$y =xv$$
    При этом $dy = vdx + x dv$
    \item Подставим замену и получим:
    $$P(x,xv) dx + Q(x,xv) (vdx + xdv)=0$$
    
    $$x^{\alpha}P(1,v)dx + x^{\alpha}Q(1,v) (vdx + xdv) =0$$
    
    $$(P(1,v) + Q(1,v)v)dx + Q(1,v)xdv =0$$
\end{enumerate}

\textbf{Уравнения, сводящиеся к однородному}

Уравнения в нормальной форме:
$$y' = f\left(\cfrac{y}{x}\right)$$
сводится к однородному при переходе к дифференциалам.

Более общее уравнение 
$$y' = f \left(\cfrac{a_1x + b_1 y + c}{a_2x + b_2 y + c_2}
\right)$$
сводится к однородному, если сдвинуть систему координат в точку пересечения прямых $a_1x  + b_1y + c_1 =0$, $a_2 x + b_2 y + c_2 = 0$. То есть если сделать замену:
$$x = x_0 +u, \quad y = y_0 + v$$

\textbf{Геом. свойство однородного уравнения} --- гомотетия относительно начала координат любую интегральную кривую однородного уравнения переводит в другую его интегральную кривую.

\subsection{Уравнение Бернулли}

\deff{def:} \textbf{Уравнением Бернулли} называют уравнение вида:
$$y' = p(x) y =+ q(x) y^{\alpha}$$
где $a \not \in \{0,1\} ,a\in \mathbb{R}$

\textbf{Давайте научимся его решать:}

Возьмем $z = y^{1-\alpha}$

Тогда $z' = (1-\alpha) y^{-\alpha}y'$, откуда
$$y^{-\alpha}y' = \cfrac{z'}{1-\alpha}$$
Поделим левую часть исходного уравнения на $y^{\alpha}$, подставляя $z = y^{1-\alpha}$, а также умножая обе части на $(1-\alpha)$ получим:
$$z' = (1-\alpha) p(x) z + (1-\alpha)q(x)$$
Таким образом замена $z = y^{1-\alpha}$ сводит уравнение Бернулли к линейному, а его мы уже умеем решать

\textbf{Уравнение Риккати}

\deff{def:}  \textbf{Уравнением Риккати} называют уравнение вида:
$$y' = p(x) y^2 + q(x) y + r(x)$$
Чтобы такое решить, надо решить правое уравнение относительное $y$ и сделать подстановку $y =  z + \varphi$. Так оно сведется к уравнению Бернулли и победится.

\thmm{Теорема (Луивилль)}

Уравнение
$$y' = y^2 + x^{\alpha}$$
интегрируется в квадратурах, если и только если $\alpha / (2\alpha + 4) \in \mathbb{Z}$ или $\alpha = -2$