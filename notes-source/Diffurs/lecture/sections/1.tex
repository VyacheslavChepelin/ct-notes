\section{Лекция 1.}

\subsection{Основные определения.}

\deff{def:} Пусть $f:  G \subset \mathbb{R}^2 \rightarrow \mathbb{R}$, \deff{нормальное уравнение} :
$$y' = f(x,y)$$
\deff{def:} \deff{Область определения} нормального уравнения --- область определения его правой части. Обозначение $dom = G$.

\textbf{Примеры уравнений} и соответствующих областей определения:

\begin{enumerate}
    \item $y' = x\sqrt{y}$, $G = \mathbb{R} \times [0, + \infty)$
    \item $y' = y, G  = \mathbb{R}^2$
    \item $y' =  -\frac{1}{x^2}, G = \{(x,y)\in \mathbb{R} | x \neq 0\}$
\end{enumerate}

\deff{def:} Функция $\varphi : E \rightarrow \mathbb{R}$ - \deff{решение} уравнения, если $E = \langle a,b\rangle$:
$$\varphi'(x) \equiv f(x, \varphi(x))$$

\deff{Соглашение:} На протяжении курса, будем считать, что $\forall$ предиката $P(x)$, который не определен при $x = x_0$, считаем, что $P(x_0) = 0$ - то есть ложно. 

\textbf{Замечание:} Данное зам. помогает не требовать от $\varphi$ дифференцируемости на всем $E$.


\textbf{Следствие:} Учитывая соглашение любое решение уравнение --- дифференцируемая функция.

\textbf{Следствие:} Если $f$ - непр. функция, то любое решение нормального уравнения непрерывно дифференцируемо.


\textbf{Замечание:} В нормальном уравнении символы $x$, $y$ и $y'$ - три различные независимые переменные. Пока не произведена подстановка функции, буква $y$ никак не связана с $x$, а $y'$ не олицетворяет производную.

\deff{def:} \deff{Интегральная кривая} уравнения --- график его решения.

\deff{def:} \deff{Общее решение} уравнения --- множество всех его решений.

\deff{def:} \deff{Общим интегралом} уравнения будем называть соотношение вида:
$$\varPhi(x,y,C) = 0$$
которое неявно задает некоторые уравнения при некоторых значениях вещественного параметра $C$.

\textbf{Замечание:} Общий интеграл не всегда описывает все решения уравнения.



\subsection{Уравнение в дифференциалах.}

\deff{def:} Пусть $P,Q: G \subset \mathbb{R}^2 \rightarrow \mathbb{R}$, \deff{уравнение в дифференциалах}:
$$P(x,y)dx +Q(x,y) dy = 0$$
\deff{Замечание:}  Переменные $x$, $y$ входят равноправно, поэтому его решением называется не только функция $y= \varphi(x)$, но и $x = \psi(y)$

\deff{def:} Точка  $(x_0,y_0) \in G$ называется \deff{особой точкой} уравнения, если $P(x_0,y_0) = Q(x_0,y_0) = 0$

\deff{def:} Пусть $T = \langle a,b\rangle$, вектор-функции $(u,v)\in \mathbb{C}^1(T\rightarrow \mathbb{R}^2)$ --- \deff{параметрические} решение уравнения, если:
\begin{enumerate}
    \item $(u'(t),v'(t)) \neq (0,0)$ для всех $t \in T$
    \item $P(u(t),v(t)) u'(t) + Q(u(t),v(t))v'(t) \equiv 0$ на $T$
\end{enumerate}

\deff{def:}  \deff{Интегральной кривой} уравнения называют годограф (множество значений) ее параметрического уравнения. 

\deff{Утверждение:} (Связь между обычными и параметрическими решениями). 

Пусть $P,Q \in C(G)$, множество $G$ не содержит особых точек уравнения, тогда:
\begin{enumerate}
    \item Если $y = \varphi(x)$ --- решение уравнения на $E$, то $r(t) = (t, \varphi(t))$ - параметрическое решение уравнения на $E$.
    \item Если $r= (u,v)$ --- параметрическое  решение уравнения на $T$, то для любого $t_0 \in T$, найдется окрестность $U(t_0)$, такая что функции $u(t)$ и $v(t)$ при $t\in U(t_0)\cap T$ параметрически задают решение уравнения.
\end{enumerate}

\deff{def:} Два дифференциальных уравнения \textbf{эквивалентны} (или \textbf{равносильны}) на множестве $G$, если они имеют одинаковое семейство интегральных кривых на множестве $G$.

\thmm{Теорема.} Пусть $f \in C(G\subset \mathbb{R}^2 \rightarrow \mathbb{R})$. Тогда уравнение:
$$y' = f(x,y)$$
эквивалентно на множестве $G$ уравнению:
$$dy = f(x,y) dx$$