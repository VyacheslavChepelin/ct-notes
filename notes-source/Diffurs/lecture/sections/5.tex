\section{Лекция 5.}

\subsection{Уравнение высшего порядка.}


\deff{def:} \textbf{Дифференциальным уравнением n-го порядка} называют уравнение вида
$$F(x,y,y', \dots,y^{(n)})=0$$ %как поставить (n) в конце строки
\deff{def:}  Функция $\varphi:E  \to  \mathbb{R}$  
- \textbf{решение  уравнения},  если  $E    =    \langle a,b \rangle$ и 
$$F(x, \varphi(x), \varphi'(x),\dots,\varphi^{(n)}(x)) \equiv 0 \text{  на  } E.$$
\deff{def:} \textbf{Каноничным уравнением} будем называть уравнение 
$$y^{(n)} =   f(x,y,y',\dots,y^{(n-1)},$$
разрешенное относительно старшей производной.



\deff{def:} \textbf{Задачей Коши} для канонического уравнения называют задачу нахождения его решения, удовлетворяющего начальным условиям:
$$y(x_0) = y_0, y'(x_0) = y_0', \ldots, y^{(n-1)}(x_0) = y_0^{(n-1)} $$
\textbf{Замечание:} в данном случае стоит воспринимать $y_0^{(i)}$ как значение, а не как производную от числа.


\subsection{Методы понижения порядка}

\begin{enumerate}
    \item \textbf{Уравнения вида $y^{(n)} = f(x)$}
    
Хм хм, что же делать? Возьмем $n$ раз интеграл - победили.
    \item  \textbf{Уравнение без искомой функции:}
    Пусть у нас есть уравнение:
    $$F(x,y'^{(k)},\ldots y^{(n)}) = 0$$
    Тогда стоит сделать замену $z(x) = y^{(k)}$

    \item \textbf{Уравнение без независимой переменной}

    Пусть у нас есть уравнение:
    $$F(y,y',y'',\ldots y^{(n)}) = 0$$
    Сделаем подстановку:
    $$y'(x) = z(y(x))$$

    \item \textbf{Уравнение, однородное относительно искомой функции и ее производных}

    Пусть при любом допустимом значении $t$:
    $$F(x,ty,ty', \ldots, ty^{(n)}) = t^m F(x,y,y', \ldots, y^{(n})$$
    Тогда порядок уравнения понижается при помощи замену $z = \cfrac{y'}{y}$
    \item  \textbf{Уравнение в точных производных}

    Если наша функция это производная какой-то другой по переменной $x$, то есть:
    $$F(x,y(x),y'(x), \ldots, y^{(n)}(x)) = \cfrac{d}{dx} \Phi(x,y(x),\ldots , y^{(n-1)}(x))$$
    то мы можем понизить порядок на 1 вниз решая:
    $$\Phi(x,y(x),\ldots , y^{(n-1)}(x)) = C$$
\end{enumerate}





\subsection{Нормальная система}

\deff{def:}  \textbf{Нормальной системой} дифференциальных уравнений  порядка n называется система вида\
$$\begin{cases}
r'_1  = f_1 (t, r_1, \dots, r_n) \\
\dots \\
r'_n   =   f_n(t, r_1, \dots, r_n)
\end{cases}$$
Если положить 
$$   r =
\begin{pmatrix}
r_1  \\
\dots \\
r_n
\end{pmatrix}, f(t,r) = 
\begin{pmatrix}
f_1 (t,r)  \\
\dots   \\
f_n(t,r)
\end{pmatrix},
$$   % короче   скобки поправишь да
то система компактно в виде одного n-мерного уравнения

$$r' =f(t,r).$$

\deff{def:}  Вектор-функция $\varphi: E \rightarrow \mathbb{R}^n$ -  \textbf{решение системы}, если $E  = \langle a,b \rangle$ и $\varphi'(t) \equiv f(t,\varphi(t))$ на $E$.

\deff{def:} \textbf{Интегральной кривой} системы называют график, соответствующий ее решению (что не удивительно)

В отличие от одномерного случая, интегральная кривая - это график вектор-функции, расположенный в (n+1)-мерном пространстве.

% чек тг



\deff{def:} \textbf{Задачей Коши} называется аналогичная уравнению высшему порядку конструкция

\deff{def:} Зададим отображение $\Lambda_n$ формулой:
$$\Lambda_n \varphi = (\varphi,  \varphi',\ldots , \varphi^{(n-1)})^T$$
Индекс $n$ будет иногда опускаться.

\thmm{Лемма. (о системе равносильной уравнению)}

Отображение  $\Lambda_n$ - биекция между решениями уравнения 
$$y^{(n)} = f(t,y,\ldots, y^{(n-1)})$$
и решениями системы:
$$r' = \begin{vmatrix}
    r_2\\
    r_3\\
    \ldots \\
    r_n\\
    f(t,r)
\end{vmatrix}$$

\textbf{Замечание:} Это лемма очевидна из того, что мы просто сопоставляем каждой производной отдельную функцию и пишем уравнение для этой производной по типу $(y')' = y'' \Leftrightarrow r_1' = r_2$


\deff{def:} Такую систему будем называть \textbf{системой равносильной уравнению}.
