\section{Лекция 3.}

\subsection{Интегрирующий множитель.}

\deff{def:} Функция $u: G \rightarrow \mathbb{R}$ называются \deff{интегрирующим множителем} уравнения $P(x,y) dx + Q(x,y) dy =0$ в области $G$, если $\mu(x,y)\neq 0$ для любой точки $(x,y) \in G$ и уравнение
$$\mu(x,y)P(x,y)dx +   \mu(x,y) Q(x,y)dy = 0$$
является уравнением в полных дифференциалах.

\textbf{Замечание:} мы хотим получить УПД и чтобы его получить, мы хотим, чтобы $P'_y = Q'_x$, для этого добавляем множитель $\mu$.


\deff{def:} Пусть $p_2(x) \neq 0$ при $x\in (a,b), q_1(y) \neq  0 $ при $y \in (c,d)$. Тогда функция
$$\mu(x,y) = \cfrac{1}{p_2(x) q_1(y)}$$
является интегрирующем множителем для уравнения:
$$p_1(x) q_1(y)dx  + p_2(x) q_2(y) dy = 0$$
Такое уравнение называются уравнением c \deff{разделяющимися переменными}.


\textbf{Условие для интегрирующего множества:} Пусть $P,Q \in C^1(G)$. Определим условия для интегрирующего множителя из $\mathbb{C}^1(G)$. Необходимо:
$$\mu'_y P - \mu_x'Q = (Q'_x - P'_y)\mu$$

\subsection{Линейное уравнение.}

\deff{def:} Дифференциальное уравнение:
$$y' = p(x) y  + q(x)$$
называется \deff{линейным уравнением} первого порядка.

\deff{def:} Линейное уравнение называется \deff{однородным}, если $q=0$, иначе уравнение называется \deff{неоднородным}.

\deff{Приведение линейного уравнения 1 порядка к УПД:}
$$y' = p(x) y + q(x)$$
Заменим $y'$ на $\cfrac{dy}{dx}$:
$$(py+q) dx - dy = 0$$
Условие $P'_y =Q'_x$ здесь не выполнено. Посмотрим на условие для интегрирующего множества. Оно принимает вид:
$$\mu'_y(pu+q) + \mu'_x = -p \mu$$
Попробуем найти интегрирующий множитель, зависящий только от переменной $x$. В этом случае получим:
$$\mu'=-p\mu$$
Одно из его решений:
$$\mu = e^{-\int p}$$
Откуда мы можем решать его, как уравнение в дифференциалах

\subsection{Уравнение с разделяющимися переменными.}

$$p_1(x)q_1(y)dx + p_2(x) q_2(y)dy = 0$$
Проблема в том, что умножая на интегрирующий множитель $\cfrac{1}{q_1(y)p_2(x)}$ возможно лишь в области, где знаменатель не обращается в ноль. Случай $q_1(y) =0$ и $p_2(x) =0$ требуют особого рассмотрения.

Разбив всю область поиска интегральных кривых на необходимое количество частей, нужно рассмотреть исходное уравнение на каждой части отдельно. На каждой такой подобласти его можно разделить на $q_1(y)p_2(x)$ не опасаясь.

Остается изучить поведение найденных интегральных кривых вблизи границы и мы победим.

\subsection{Линейное уравнение первого порядка}

\thmm{Теорема (общее решение ЛУ 1-го порядка)}

Пусть $E = \langle a,b\rangle,p,q\in C(E), \mu = e^{-\int p}$. Тогда общее решение имеет вид:
$$y = \cfrac{C+ \int q\mu}{\mu}, C \in \mathbb{R}$$

\textbf{Доказательство:}

После приведения к УПД, получаем:
$$y'e^{-\int p}-p y e^{- \int p}= qe^{-\int p}$$
Левая часть - производная $y$ и $e^{-\int p}$. Получаем:
$$(y e^{-\int p})' = q e^{-\int p} $$
Следовательно:
$$ye^{-\int p} = C + \int qe^{-\int p}$$
\hfill Q.E.D.


\textbf{Следствие (Общее решение ЛОУ первого порядка):}

Пусть $E = \langle a,b \rangle$, $p \in C(E)$. Тогда уравнение $$y'=p(x)y$$
имеет вид:
$$y = C e^{\int p}, C\in \mathbb{R}, x\in E$$

\deff{Метод Лагранжа.}

\begin{enumerate}
    \item Решим вспомогательное уравнение $y' = p(x)y$
    \item Заменим в решении $C$ на $C(x)$
    \item Подставим полученное $\varphi$ в исходное уравнение и найдем $C(x)$
    \item Победа!
\end{enumerate}
